% \iffalse meta-comment
%
% Copyright (c) 2010 by Martin Scharrer <martin@scharrer-online.de>
% -----------------------------------------------------------------
%
% This work may be distributed and/or modified under the
% conditions of the LaTeX Project Public License, either version 1.3c
% of this license or (at your option) any later version.
% The latest version of this license is in
%
%   http://www.latex-project.org/lppl.txt
%
% and version 1.3c or later is part of all distributions of LaTeX
% version 2008/05/04 or later.
%
% This work has the LPPL maintenance status `maintained'.
%
% The Current Maintainer of this work is Martin Scharrer.
%
% This work consists of the files standalone.dtx, standalone.ins
% and the derived file standalone.sty.
%
% $Id$
% \fi
%
% \iffalse
%<cls|sty|cfg|tex>% $Id$
%<cls|sty>\NeedsTeXFormat{LaTeX2e}
%<cls>\ProvidesClass{standalone}   [2010/03/22 v0.2 Class to compile TeX sub-files standalone]
%<sty>\ProvidesPackage{standalone} [2010/03/22 v0.2 Package to include TeX sub-files which use the 'standalone' class]
%<tex>\ProvidesFile{standalone.tex}[2010/03/22 v0.2 Provides if-switch to show if file is compiled standalone]%
%<cfg>\ProvidesFile{standalone.cfg}[2010/03/22 v0.2 Default configuration file for 'standalone' class and package]%
%<*driver>
\NeedsTeXFormat{LaTeX2e}
\ProvidesFile{standalone.dtx}[2010/03/22 v0.2 DTX File for 'standalone' cls/sty/tex/cfg files]
\documentclass{ltxdoc}
\usepackage{ifpdf}
\usepackage{listings}
\IfFileExists{hypdoc.sty}{%
\usepackage{hypdoc}
}{%
\usepackage{hyperref}
}

\EnableCrossrefs
\CodelineIndex
\RecordChanges
%\OnlyDescription
\providecommand{\env}[1]{\texttt{#1}}
\begin{document}
  \DocInput{\jobname.dtx}
  \PrintChanges
  \newpage\PrintIndex
\end{document}
%</driver>
% \fi
%
% \CheckSum{242}
%
% \CharacterTable
%  {Upper-case    \A\B\C\D\E\F\G\H\I\J\K\L\M\N\O\P\Q\R\S\T\U\V\W\X\Y\Z
%   Lower-case    \a\b\c\d\e\f\g\h\i\j\k\l\m\n\o\p\q\r\s\t\u\v\w\x\y\z
%   Digits        \0\1\2\3\4\5\6\7\8\9
%   Exclamation   \!     Double quote  \"     Hash (number) \#
%   Dollar        \$     Percent       \%     Ampersand     \&
%   Acute accent  \'     Left paren    \(     Right paren   \)
%   Asterisk      \*     Plus          \+     Comma         \,
%   Minus         \-     Point         \.     Solidus       \/
%   Colon         \:     Semicolon     \;     Less than     \<
%   Equals        \=     Greater than  \>     Question mark \?
%   Commercial at \@     Left bracket  \[     Backslash     \\
%   Right bracket \]     Circumflex    \^     Underscore    \_
%   Grave accent  \`     Left brace    \{     Vertical bar  \|
%   Right brace   \}     Tilde         \~}
%
%
% \changes{v0.1}{2010/03/21}{First released version}
% \changes{v0.2}{2010/03/22}{Added support for beamer style settings in the config file using \cs{ifstandalonebeamer}. Updated documentation.}
%
% \GetFileInfo{\jobname.dtx}
%
% \DoNotIndex{\newcommand,\newenvironment,\def,\edef,\xdef,\DeclareRobustCommand}
% \DoNotIndex{\expandafter,\if,\else,\fi,\ifnum,\ifx,\let,\global,\long}
% \DoNotIndex{\newcounter,\newcount,\message,\meaning,\noexpand,\relax,\value}
% \DoNotIndex{\setcounter,\addtocounter,\advance,\afterassignment,\AtEndOfPackage}
% \DoNotIndex{\ProvidesPackage,\providecommand,\RequirePackage,\empty,\begin,\end}
% \DoNotIndex{\begingroup,\bgroup,\egroup,\endgroup,\csname,\endcsname,\@tempa,\@tempb}
% \DoNotIndex{\ignorespaces,\lccode,\sffamily,\@gobble,\@ifundefined,\@for,\or}
% \DoNotIndex{\@firstoftwo,\@ifnextchar,\@namedef,\@nameuse,\@secondoftwo}
% \DoNotIndex{\@temptokena,\toks@,\BODY,\do,\g@addto@macro,\lowercase,\uppercase,\the}
% \DoNotIndex{\aftergroup,\CurrentOption,\DeclareBoolOption,\DeclareDefaultOption,\DeclareStringOption,\DeclareVoidOption}
% \DoNotIndex{\endinput,\endpreview,\input,\LoadClass,\newif,\PassOptionsToPackage,\preview,\ProcessKeyvalOptions}
% \DoNotIndex{\renewcommand,\renewenvironment,\setbeamertemplate,\SetupKeyvalOptions,\space,\unskip,\usepackage}
%
% \ifpdf
% \hypersetup{%
%   pdfauthor   = {Martin Scharrer <martin@scharrer-online.de>},
%   pdftitle    = {The standalone class and package},
%   pdfsubject  = {Documentation of LaTeX class and package 'standalone'},
%   pdfkeywords = {standalone, LaTeX, TeX}
% }%
% \fi
% \clearpage
% \null
% \vspace*{-2em}
% \begin{center}
%   {\LARGE\sffamily The \emph{standalone} Class and Package\\[\medskipamount]}
%   {\large Martin Scharrer \\[\medskipamount]\normalsize 
%   \url{martin@scharrer-online.de}\\[.8ex]
%   \url{http://www.ctan.org/pkg/standalone/}\\[\bigskipamount]}
%   {\large Version \fileversion\ -- \filedate}\\
% \end{center}
% \vspace{1.2em}%
%
% \section{Introduction}
% Larger \LaTeX{} documents can be split into multiple \TeX\ files which are then included in a main document with \cs{include}
% for e.g.\ chapter files or \cs{input} for e.g.\ \TeX-coded pictures. Keeping pictures in their own sub-files improves readability 
% of the main file and simplifies the sharing of them between different documents. However, during the, sometimes lengthly, 
% drawing/coding process it has benefits to be able to compile the pictures on their own. The compile process is much quicker and
% the resulting document only holds the picture which avoids constant page turning and zooming.
%
% While it is possible to write a small `main' file for each picture file, this method is a little cumbersome and clutters
% the directories with a lot of extra files. A second method is to place the `main' components, i.e. a preamble, directly into the
% picture files and make the main document ignore this code sections.
%
% The package |standalone| can be used in the main document to skip all extra preambles in included files.
%
% A |standalone| class is also provided to minimise the extra preamble code needed in this files.
% It's usage is optional, but simplifies and standardises how picture files are compiled standalone.
% The class uses by default the |preview| package to create an output file which only contains the picture with no extra
% margins, page numbers or anything else. A configuration file |standalone.cfg| read by the class allows the user to adjust settings and macros easily 
% on a per directory base.
%
% \subsection*{Similar Packages}
% The |docmute| package by T.M.\ Trzeciak is written with the same function as the |standalone| package. However, no special class file or 
% configuration file is provided.
%
% \section{Usage}
% \subsection{Quick instructions}
% Load the |standalone| \emph{package} and all packages needed by all the sub-files in the main document and include your picture or other sub-files using |\input| as normal.
% In the sub-files use the |standalone| \emph{class} using the normal \cs{documentclass} and load all packages needed for the particular 
% file. Finally wrap the actual file content in a |document| environment.
%
% When the sub-file is compiled on its own the |\documentclass| and |document| environment will be active as normal. The main file, however,
% will skip everything from the |\documentclass| till the |\begin{document}|. The (now fake) |document| environment is redefined to be a simple 
% TeX-group. Any code after the |\end{document}| will be ignored. The real |document| environment of the main file will be unaffected and will work as normal.
%
% \subsection{Class Options}
% The |standalone| class will load a real document class. By default this is |article|. The document class normally has not much
% influence on sub-files like pictures, especially when the |preview| package is active. However, the used class can be adjusted by the user
% with the |class=|\meta{class name} option.
%
% A special |beamer| option is provided to handle beamer frames and overlays correctly. See section~\ref{sec:beamer} for more information.
%
% All other used options are passed to the loaded class.
%
% \subsection{Package Options}
% At the moment the |standalone| package does not provide any options.
%
% \subsection{Environments and Macros}
%
% \DescribeEnv{standalone}
% The \env{standalone} environment is wrapped around the content of each sub-file when compiled standalone.
% By default it only contains a \env{preview} environment as long the class is not called with the |preview=false| option.
% It can be redefined in the configuration file if required.
% When compiled as part of a main document the \env{standalone} environment does nothing (apart of being a \TeX\ group).
%
% \DescribeEnv{standaloneframe}
% The \env{standaloneframe} environment is only defined when the class is called with the |beamer| option and acts as a replacement
% of the \env{frame} environment of beamer when compiled standalone. All optional arguments of \env{frame} are supported.
% When compiled as part of a main document it does nothing except of gobbling its arguments.
%
% \bigskip
% \DescribeMacro{\ifstandalone}
% Both the class and the package provide the if-switch \cs{ifstandalone}, which can be used to only include code if the file is 
% compiled standalone. The switch is set to \cs{iftrue} by the class and to \cs{iffalse} by the package.
%
% The additional file |standalone.tex| also defines this switch by checking if \cs{documentclass} was already used. It can be included
% with |% \iffalse meta-comment
%
% Copyright (c) 2010 by Martin Scharrer <martin@scharrer-online.de>
% -----------------------------------------------------------------
%
% This work may be distributed and/or modified under the
% conditions of the LaTeX Project Public License, either version 1.3c
% of this license or (at your option) any later version.
% The latest version of this license is in
%
%   http://www.latex-project.org/lppl.txt
%
% and version 1.3c or later is part of all distributions of LaTeX
% version 2008/05/04 or later.
%
% This work has the LPPL maintenance status `maintained'.
%
% The Current Maintainer of this work is Martin Scharrer.
%
% This work consists of the files standalone.dtx, standalone.ins
% and the derived file standalone.sty.
%
% $Id$
% \fi
%
% \iffalse
%<cls|sty|cfg|tex>% $Id$
%<cls|sty>\NeedsTeXFormat{LaTeX2e}
%<cls>\ProvidesClass{standalone}   [2010/03/20 v0.1 Class to compile TeX sub-files standalone]
%<sty>\ProvidesPackage{standalone} [2010/03/20 v0.1 Package to include TeX sub-files which use the 'standalone' class]
%<tex>\ProvidesFile{standalone.tex}[2010/03/20 v0.1 Provides if-switch to show if file is compiled standalone]%
%<cfg>\ProvidesFile{standalone.cfg}[2010/03/20 v0.1 Default configuration file for 'standalone' class and package]%
%<*driver>
\NeedsTeXFormat{LaTeX2e}
\ProvidesFile{standalone.dtx}[2010/03/20 v0.1 DTX File for 'standalone' cls/sty/tex]
\documentclass{ltxdoc}
\usepackage{ifpdf}
\usepackage{hyperref}
\usepackage{tabularx}
\usepackage{array}
\usepackage{flafter,fnpos}
\usepackage{booktabs}
\usepackage{amsmath}
\usepackage{placeins}
\makeFNbottom
\makeFNbelow

\EnableCrossrefs
%\CodelineIndex
\RecordChanges
%\OnlyDescription
\begin{document}
  \DocInput{\jobname.dtx}
  \PrintChanges
  %\newpage\PrintIndex
\end{document}
%</driver>
% \fi
%
% \CheckSum{0}
%
% \CharacterTable
%  {Upper-case    \A\B\C\D\E\F\G\H\I\J\K\L\M\N\O\P\Q\R\S\T\U\V\W\X\Y\Z
%   Lower-case    \a\b\c\d\e\f\g\h\i\j\k\l\m\n\o\p\q\r\s\t\u\v\w\x\y\z
%   Digits        \0\1\2\3\4\5\6\7\8\9
%   Exclamation   \!     Double quote  \"     Hash (number) \#
%   Dollar        \$     Percent       \%     Ampersand     \&
%   Acute accent  \'     Left paren    \(     Right paren   \)
%   Asterisk      \*     Plus          \+     Comma         \,
%   Minus         \-     Point         \.     Solidus       \/
%   Colon         \:     Semicolon     \;     Less than     \<
%   Equals        \=     Greater than  \>     Question mark \?
%   Commercial at \@     Left bracket  \[     Backslash     \\
%   Right bracket \]     Circumflex    \^     Underscore    \_
%   Grave accent  \`     Left brace    \{     Vertical bar  \|
%   Right brace   \}     Tilde         \~}
%
%
% \changes{v0.1}{2010/03/20}{First released version}
%
% \GetFileInfo{\jobname.dtx}
%
% \DoNotIndex{\newcommand,\newenvironment,\def,\edef,\xdef,\DeclareRobustCommand}
% \DoNotIndex{\expandafter,\if,\else,\fi,\ifnum,\ifx,\let,\global,\long}
% \DoNotIndex{\newcounter,\newcount,\message,\meaning,\noexpand,\relax,\value}
% \DoNotIndex{\setcounter,\addtocounter,\advance,\afterassignment,\AtEndOfPackage}
% \DoNotIndex{\ProvidesPackage,\providecommand,\RequirePackage,\empty,\begin,\end}
% \DoNotIndex{\begingroup,\bgroup,\egroup,\endgroup,\csname,\endcsname,\@tempa,\@tempb}
% \DoNotIndex{\ignorespaces,\lccode,\sffamily,\@gobble,\@ifundefined,\@for,\or}
% \DoNotIndex{\@firstoftwo,\@ifnextchar,\@namedef,\@nameuse,\@secondoftwo}
% \DoNotIndex{\@temptokena,\toks@,\BODY,\do,\g@addto@macro,\lowercase,\uppercase,\the}
%
% \ifpdf
% \hypersetup{%
%   pdfauthor   = {Martin Scharrer <martin@scharrer-online.de>},
%   pdftitle    = {The standalone class and package},
%   pdfsubject  = {Documentation of LaTeX class and package 'standalone'},
%   pdfkeywords = {standalone, LaTeX, TeX}
% }%
% \fi
% \clearpage
% \null
% \vspace*{-2em}
% \begin{center}
%   {\LARGE\sffamily The \emph{standalone} Class and Package\\}
%   {\large Martin Scharrer \\\normalsize 
%   \url{martin@scharrer-online.de}\\[.8ex]
%   \url{http://www.ctan.org/pkg/standalone/}\\[1.2ex]}
%   {\large Version \fileversion\ -- \filedate}\\
% \end{center}
% \vspace{1.2em}%
%
% \section{Introduction}
%
% \StopEventually{}
% \FloatBarrier
% \clearpage
%
% \section{Implementation}
%
% \subsection{The Package File}
% \iffalse
%<*sty>
% \fi
%    \begin{macrocode}
\expandafter\newif\csname ifstandalone\endcsname
\standalonefalse
%    \end{macrocode}
%
%  \begin{macro}{\sa@documentclass}
%    \begin{macrocode}
\newcommand{\sa@documentclass}[2][]{%
  \let\document\sa@document
  \let\enddocument\sa@enddocument
  \expandafter\sa@@documentclass
}
%    \end{macrocode}
%  \end{macro}
%
%  \begin{macro}{\sa@@documentclass}
%    \begin{macrocode}
\def\sa@@documentclass{%
  \begingroup\def\sa@gobbleto{document}\sa@gobble
}
%    \end{macrocode}
%  \end{macro}
%
%  \begin{macro}{\sa@gobble}
%    \begin{macrocode}
\long\def\sa@gobble#1\begin#2{%
  \def\@tempa{#2}%
  \ifx\@tempa\sa@gobbleto
    \def\next{\expandafter\endgroup\expandafter\begin\expandafter{\sa@gobbleto}}%
  \else
    \def\next{\sa@gobble}%
  \fi
  \next
}
%    \end{macrocode}
%  \end{macro}
%
%    \begin{macrocode}
\expandafter\ifx\csname standalone\endcsname\relax
  \newenvironment{standalone}[1][]{}{}
\fi
%    \end{macrocode}
%
%  \begin{macro}{\sa@orig@document}
%    \begin{macrocode}
\let\sa@orig@document\document
%  \end{macro}
%
%  \begin{macro}{\sa@orig@enddocument}
%    \begin{macrocode}
\let\sa@orig@enddocument\enddocument
%    \end{macrocode}
%  \end{macro}
%
%  \begin{macro}{\document}
%    \begin{macrocode}
\def\document{%
  \sa@orig@document
  \let\documentclass\sa@documentclass
  \ignorespaces
}
%    \end{macrocode}
%  \end{macro}
%
%  \begin{macro}{\sa@document}
%    \begin{macrocode}
\def\sa@document{%
  \sa@atbegindocument
}
%    \end{macrocode}
%  \end{macro}
%
%  \begin{macro}{\sa@enddocument}
%    \begin{macrocode}
\def\sa@enddocument{%
  \sa@atenddocument
  \aftergroup\sa@@enddocument
}
%    \end{macrocode}
%  \end{macro}
%
%  \begin{macro}{\sa@atbegindocument}
%    \begin{macrocode}
\def\sa@atbegindocument{%
  \ignorespaces
}%
%    \end{macrocode}
%  \end{macro}
%
%  \begin{macro}{\sa@atenddocument}
%    \begin{macrocode}
\def\sa@atenddocument{%
  \unskip
}%
%    \end{macrocode}
%  \end{macro}
%
%  \begin{macro}{\sa@@enddocument}
%    \begin{macrocode}
\def\sa@@enddocument{%
  %\let\document\sa@orig@document
  \let\enddocument\sa@orig@enddocument
  \endinput
}
%    \end{macrocode}
%  \end{macro}
%
%  \begin{macro}{\sa@processpreamble}
%    \begin{macrocode}
\def\sa@processpreamble{%
  \renewcommand\usepackage[2][]{%
    \message{^^J%
      INFO: Sub-file requires the following package(s):^^J
      \space\space[##1]{##2}^^J%
    }%
  }%
  \let\RequirePackage\usepackage
}
%    \end{macrocode}
%  \end{macro}
%
%    \begin{macrocode}
%%\def\sa@@documentclass{\sa@processpreamble}
%    \end{macrocode}
%
% \iffalse
%</sty>
% \fi
%
% \subsection{The Class File}
% \iffalse
%<*cls>
% \fi
%    \begin{macrocode}
\def\sa@classoptionslist{}
\RequirePackage{kvoptions}
\SetupKeyvalOptions{prefix=sa@}
\DeclareBoolOption[true]{preview}
%    \end{macrocode}
%
%  \begin{macro}{\sa@cls@document}
%    \begin{macrocode}
\def\sa@cls@document{\ifsa@preview\preview\fi}
%    \end{macrocode}
%  \end{macro}
%
%  \begin{macro}{\sa@cls@enddocument}
%    \begin{macrocode}
\def\sa@cls@enddocument{\ifsa@preview\endpreview\fi}
%    \end{macrocode}
%  \end{macro}
%
%    \begin{macrocode}
\DeclareVoidOption{beamer}{%
  \def\sa@class{beamer}%
  \sa@previewfalse
  \newenvironment{standalone}[1][]{%
  \begin{frame}[fragile,environment=standalone]}{\end{frame}}%
}
\DeclareStringOption[article]{class}
\DeclareStringOption[]{classoptions}
\DeclareStringOption[]{frameoptions}
\DeclareDefaultOption{%
  \xdef\sa@classoptionslist{\sa@classoptionslist,\CurrentOption}%
}
% vim: ft=tex
\PassOptionsToPackage{active}{preview}

\ProcessKeyvalOptions*\relax
%    \end{macrocode}
%
%    \begin{macrocode}
\let\@classoptionslist\sa@classoptionslist
\xdef\@tempa{[\sa@classoptions]{\sa@class}}
\expandafter\LoadClass\@tempa
%    \end{macrocode}
%
%    \begin{macrocode}
\ifsa@preview
  \RequirePackage{preview}
\fi
%    \end{macrocode}
%
%    \begin{macrocode}
\RequirePackage{standalone}[2010/03/20]
\standalonetrue
%    \end{macrocode}
%
%  \begin{macro}{\document}
%    \begin{macrocode}
\def\document{%
  \sa@orig@document
  \let\documentclass\sa@documentclass
  \sa@cls@document
}
%    \end{macrocode}
%  \end{macro}
%
%  \begin{macro}{\enddocument}
%    \begin{macrocode}
\def\enddocument{%
  \sa@cls@enddocument
  \sa@orig@enddocument
}
%    \end{macrocode}
%  \end{macro}
%
% \iffalse
%</cls>
% \fi
%
% \subsection{Simple TeX File}
% \iffalse
%<*tex>
% \fi
%    \begin{macrocode}
\expandafter\ifx\csname ifstandalone\endcsname\relax
%    \end{macrocode}
%    \begin{macrocode}
\expandafter\newif\csname ifstandalone\endcsname
%    \end{macrocode}
%    \begin{macrocode}
\expandafter\ifx\csname @twoclasseserror\endcsname\documentclass
\else
  \standalonetrue
\fi
%    \end{macrocode}
%    \begin{macrocode}
\fi
%    \end{macrocode}
% \iffalse
%</tex>
% \fi
%
% \subsection{Config File}
% \iffalse
%<*cfg>
% \fi
%    \begin{macrocode}
\PassOptionsToPackage{active,tightpage}{preview}%
%    \end{macrocode}
% \iffalse
%</cfg>
% \fi
%
% \iffalse
%<cfg>% vim: ft=tex
% \fi
% \Finale
% \endinput
| and is intended for specialised files which do not use the |standalone| class.
%
% \def\lstlistingname{Example}
% \begin{lstlisting}[language={[latex]tex},gobble=4,frame=lines,caption={Use of \emph{standalone} class.}]
%   % A sub-file (e.g. picture) using the 'standalone' class:
%   % Use 'standalone' as document class:
%   \documentclass{standalone}
%
%   % Load packages needed for this TeX file:
%   \usepackage{tikz}
%
%   % Surround TeX code with 'document' environment as usually:
%   \begin{document}
%   % Add your TeX code, e.g. a picture:
%   \begin{tikzpicture}
%     \draw (0,0) rectangle (2,1) node [midway] {Example};
%   \end{tikzpicture}
%   \end{document}
% \end{lstlisting}
%
% \begin{lstlisting}[language={[latex]tex},gobble=4,frame=lines,caption={Effective code if compiled standalone.}]
%   \documentclass{article}
%
%   \newenvironment{standalone}{\begin{preview}}{\end{preview}}
%   % vim: ft=tex
\PassOptionsToPackage{active}{preview}

%   % which by defaults loads: \PassOptionsToPackage{active,tightpage}{preview}
%   \usepackage{preview}
%
%   \usepackage{tikz}
%
%   \begin{document}
%   \begin{standalone}
%   \begin{tikzpicture}
%     \draw (0,0) rectangle (2,1) node [midway] {Example};
%   \end{tikzpicture}
%   \end{standalone}
%   \end{document}
% \end{lstlisting}
%
% \begin{lstlisting}[language={[latex]tex},gobble=4,frame=lines,caption={Effective code if included in a main document.}]
%   \begingroup
%   \begin{tikzpicture}
%     \draw (0,0) rectangle (2,1) node [midway] {Example};
%   \end{tikzpicture}
%   \endgroup
%   \endinput
% \end{lstlisting}
%
%
% \subsection{Usage of the package}
% \begin{lstlisting}[language={[latex]tex},gobble=4,frame=lines,caption={Use of \emph{standalone} package.}]
%   % Main file
%   % Real document class:
%   \documentclass{article}
%
%   % Use the 'standalone' package:
%   \usepackage{standalone}
%
%   % Load all packages needed for all sub-files:
%   \usepackage{tikz}
%
%   % Inside the real 'document' environment read the sub-file with '\input'
%   \begin{document}
%   % ...
%   \begin{figure}
%     \input{subfile}
%     \caption{A subfile}
%   \end{figure}
%   % ...
%   \end{document}
% \end{lstlisting}
%
%
% \subsection{Support for Beamer Presentations}\label{sec:beamer}
% Presentation can be written in \LaTeX\ using the |beamer| class. Each presentation frame is wrapped in a |frame| environment. Overlay effects
% can be added using special macros. This effects result in multiple pages per frame. Pictures with such overlay effects can not be
% compiled standalone using the normal settings. Instead the |standalone| class must load the |beamer| class and wrap the content also in
% a |frame| environment while skipping the |preview| environment. To activate this settings load the |standalone| class with the |beamer| option.
% Because the |frame| environment is quite special (it normally collects all it's content and calls the \cs{frame}) and must also support
% verbatim content it is not easily possible to redefined the |document| environment to include |frame|. Also |frame| accepts options which |document| doesn't.
% Therefore a second environment called |standaloneframe| is used in the beamer picture files. It will be equal to |frame| in standalone mode, but
% without effect otherwise.
%
% \DescribeMacro{\ifstandalonebeamer}
% This if switch is only true if the class is compiled with the |beamer| option. The package sets it to false.
% It can be used to place beamer specific options in the configuration files, which should be skipped for non-beamer standalone files.
%
% \begin{lstlisting}[language={[latex]tex},gobble=4,frame=lines,caption={Use of \emph{standalone} class.}]
%   % Use of 'standalone' class with a beamer overlay:
%   \documentclass[beamer]{standalone}
%
%   % Load packages needed for this TeX file:
%   \usepackage{tikz}
%
%   % Surround TeX code with 'document' environment as usually:
%   \begin{document}
%   \begin{standaloneframe}[options, e.g. 'fragile' for verbatim content]
%    % Add your TeX code:
%     \only<1>{ One }%
%     \only<2>{ Two }%
%   \end{standaloneframe}
%   \end{document}
% \end{lstlisting}
%
% \begin{lstlisting}[language={[latex]tex},gobble=4,frame=lines,caption={Effective beamer code if compiled standalone.}]
%   \documentclass{beamer}
%
%   % vim: ft=tex
\PassOptionsToPackage{active}{preview}

%
%   \usepackage{tikz}
%
%   \begin{document}
%   \begin{frame}[your options]
%     \only<1>{ One }%
%     \only<2>{ Two }%
%   \end{frame}
%   \end{document}
% \end{lstlisting}
%
% \begin{lstlisting}[language={[latex]tex},gobble=4,frame=lines,caption={Effective code if included in a beamer presentation.}]
%   \begingroup
%     \only<1>{ One }%
%     \only<2>{ Two }%
%   \endgroup
%   \endinput
% \end{lstlisting}
%
% \subsection{\texttt{standalone.tex}}
% \begin{lstlisting}[language={[latex]tex},gobble=4,frame=lines,caption={Usage of 'standalone.tex'.}]
%   % \iffalse meta-comment
%
% Copyright (c) 2010 by Martin Scharrer <martin@scharrer-online.de>
% -----------------------------------------------------------------
%
% This work may be distributed and/or modified under the
% conditions of the LaTeX Project Public License, either version 1.3c
% of this license or (at your option) any later version.
% The latest version of this license is in
%
%   http://www.latex-project.org/lppl.txt
%
% and version 1.3c or later is part of all distributions of LaTeX
% version 2008/05/04 or later.
%
% This work has the LPPL maintenance status `maintained'.
%
% The Current Maintainer of this work is Martin Scharrer.
%
% This work consists of the files standalone.dtx, standalone.ins
% and the derived file standalone.sty.
%
% $Id$
% \fi
%
% \iffalse
%<cls|sty|cfg|tex>% $Id$
%<cls|sty>\NeedsTeXFormat{LaTeX2e}
%<cls>\ProvidesClass{standalone}   [2010/03/20 v0.1 Class to compile TeX sub-files standalone]
%<sty>\ProvidesPackage{standalone} [2010/03/20 v0.1 Package to include TeX sub-files which use the 'standalone' class]
%<tex>\ProvidesFile{standalone.tex}[2010/03/20 v0.1 Provides if-switch to show if file is compiled standalone]%
%<cfg>\ProvidesFile{standalone.cfg}[2010/03/20 v0.1 Default configuration file for 'standalone' class and package]%
%<*driver>
\NeedsTeXFormat{LaTeX2e}
\ProvidesFile{standalone.dtx}[2010/03/20 v0.1 DTX File for 'standalone' cls/sty/tex]
\documentclass{ltxdoc}
\usepackage{ifpdf}
\usepackage{hyperref}
\usepackage{tabularx}
\usepackage{array}
\usepackage{flafter,fnpos}
\usepackage{booktabs}
\usepackage{amsmath}
\usepackage{placeins}
\makeFNbottom
\makeFNbelow

\EnableCrossrefs
%\CodelineIndex
\RecordChanges
%\OnlyDescription
\begin{document}
  \DocInput{\jobname.dtx}
  \PrintChanges
  %\newpage\PrintIndex
\end{document}
%</driver>
% \fi
%
% \CheckSum{0}
%
% \CharacterTable
%  {Upper-case    \A\B\C\D\E\F\G\H\I\J\K\L\M\N\O\P\Q\R\S\T\U\V\W\X\Y\Z
%   Lower-case    \a\b\c\d\e\f\g\h\i\j\k\l\m\n\o\p\q\r\s\t\u\v\w\x\y\z
%   Digits        \0\1\2\3\4\5\6\7\8\9
%   Exclamation   \!     Double quote  \"     Hash (number) \#
%   Dollar        \$     Percent       \%     Ampersand     \&
%   Acute accent  \'     Left paren    \(     Right paren   \)
%   Asterisk      \*     Plus          \+     Comma         \,
%   Minus         \-     Point         \.     Solidus       \/
%   Colon         \:     Semicolon     \;     Less than     \<
%   Equals        \=     Greater than  \>     Question mark \?
%   Commercial at \@     Left bracket  \[     Backslash     \\
%   Right bracket \]     Circumflex    \^     Underscore    \_
%   Grave accent  \`     Left brace    \{     Vertical bar  \|
%   Right brace   \}     Tilde         \~}
%
%
% \changes{v0.1}{2010/03/20}{First released version}
%
% \GetFileInfo{\jobname.dtx}
%
% \DoNotIndex{\newcommand,\newenvironment,\def,\edef,\xdef,\DeclareRobustCommand}
% \DoNotIndex{\expandafter,\if,\else,\fi,\ifnum,\ifx,\let,\global,\long}
% \DoNotIndex{\newcounter,\newcount,\message,\meaning,\noexpand,\relax,\value}
% \DoNotIndex{\setcounter,\addtocounter,\advance,\afterassignment,\AtEndOfPackage}
% \DoNotIndex{\ProvidesPackage,\providecommand,\RequirePackage,\empty,\begin,\end}
% \DoNotIndex{\begingroup,\bgroup,\egroup,\endgroup,\csname,\endcsname,\@tempa,\@tempb}
% \DoNotIndex{\ignorespaces,\lccode,\sffamily,\@gobble,\@ifundefined,\@for,\or}
% \DoNotIndex{\@firstoftwo,\@ifnextchar,\@namedef,\@nameuse,\@secondoftwo}
% \DoNotIndex{\@temptokena,\toks@,\BODY,\do,\g@addto@macro,\lowercase,\uppercase,\the}
%
% \ifpdf
% \hypersetup{%
%   pdfauthor   = {Martin Scharrer <martin@scharrer-online.de>},
%   pdftitle    = {The standalone class and package},
%   pdfsubject  = {Documentation of LaTeX class and package 'standalone'},
%   pdfkeywords = {standalone, LaTeX, TeX}
% }%
% \fi
% \clearpage
% \null
% \vspace*{-2em}
% \begin{center}
%   {\LARGE\sffamily The \emph{standalone} Class and Package\\}
%   {\large Martin Scharrer \\\normalsize 
%   \url{martin@scharrer-online.de}\\[.8ex]
%   \url{http://www.ctan.org/pkg/standalone/}\\[1.2ex]}
%   {\large Version \fileversion\ -- \filedate}\\
% \end{center}
% \vspace{1.2em}%
%
% \section{Introduction}
%
% \StopEventually{}
% \FloatBarrier
% \clearpage
%
% \section{Implementation}
%
% \subsection{The Package File}
% \iffalse
%<*sty>
% \fi
%    \begin{macrocode}
\expandafter\newif\csname ifstandalone\endcsname
\standalonefalse
%    \end{macrocode}
%
%  \begin{macro}{\sa@documentclass}
%    \begin{macrocode}
\newcommand{\sa@documentclass}[2][]{%
  \let\document\sa@document
  \let\enddocument\sa@enddocument
  \expandafter\sa@@documentclass
}
%    \end{macrocode}
%  \end{macro}
%
%  \begin{macro}{\sa@@documentclass}
%    \begin{macrocode}
\def\sa@@documentclass{%
  \begingroup\def\sa@gobbleto{document}\sa@gobble
}
%    \end{macrocode}
%  \end{macro}
%
%  \begin{macro}{\sa@gobble}
%    \begin{macrocode}
\long\def\sa@gobble#1\begin#2{%
  \def\@tempa{#2}%
  \ifx\@tempa\sa@gobbleto
    \def\next{\expandafter\endgroup\expandafter\begin\expandafter{\sa@gobbleto}}%
  \else
    \def\next{\sa@gobble}%
  \fi
  \next
}
%    \end{macrocode}
%  \end{macro}
%
%    \begin{macrocode}
\expandafter\ifx\csname standalone\endcsname\relax
  \newenvironment{standalone}[1][]{}{}
\fi
%    \end{macrocode}
%
%  \begin{macro}{\sa@orig@document}
%    \begin{macrocode}
\let\sa@orig@document\document
%  \end{macro}
%
%  \begin{macro}{\sa@orig@enddocument}
%    \begin{macrocode}
\let\sa@orig@enddocument\enddocument
%    \end{macrocode}
%  \end{macro}
%
%  \begin{macro}{\document}
%    \begin{macrocode}
\def\document{%
  \sa@orig@document
  \let\documentclass\sa@documentclass
  \ignorespaces
}
%    \end{macrocode}
%  \end{macro}
%
%  \begin{macro}{\sa@document}
%    \begin{macrocode}
\def\sa@document{%
  \sa@atbegindocument
}
%    \end{macrocode}
%  \end{macro}
%
%  \begin{macro}{\sa@enddocument}
%    \begin{macrocode}
\def\sa@enddocument{%
  \sa@atenddocument
  \aftergroup\sa@@enddocument
}
%    \end{macrocode}
%  \end{macro}
%
%  \begin{macro}{\sa@atbegindocument}
%    \begin{macrocode}
\def\sa@atbegindocument{%
  \ignorespaces
}%
%    \end{macrocode}
%  \end{macro}
%
%  \begin{macro}{\sa@atenddocument}
%    \begin{macrocode}
\def\sa@atenddocument{%
  \unskip
}%
%    \end{macrocode}
%  \end{macro}
%
%  \begin{macro}{\sa@@enddocument}
%    \begin{macrocode}
\def\sa@@enddocument{%
  %\let\document\sa@orig@document
  \let\enddocument\sa@orig@enddocument
  \endinput
}
%    \end{macrocode}
%  \end{macro}
%
%  \begin{macro}{\sa@processpreamble}
%    \begin{macrocode}
\def\sa@processpreamble{%
  \renewcommand\usepackage[2][]{%
    \message{^^J%
      INFO: Sub-file requires the following package(s):^^J
      \space\space[##1]{##2}^^J%
    }%
  }%
  \let\RequirePackage\usepackage
}
%    \end{macrocode}
%  \end{macro}
%
%    \begin{macrocode}
%%\def\sa@@documentclass{\sa@processpreamble}
%    \end{macrocode}
%
% \iffalse
%</sty>
% \fi
%
% \subsection{The Class File}
% \iffalse
%<*cls>
% \fi
%    \begin{macrocode}
\def\sa@classoptionslist{}
\RequirePackage{kvoptions}
\SetupKeyvalOptions{prefix=sa@}
\DeclareBoolOption[true]{preview}
%    \end{macrocode}
%
%  \begin{macro}{\sa@cls@document}
%    \begin{macrocode}
\def\sa@cls@document{\ifsa@preview\preview\fi}
%    \end{macrocode}
%  \end{macro}
%
%  \begin{macro}{\sa@cls@enddocument}
%    \begin{macrocode}
\def\sa@cls@enddocument{\ifsa@preview\endpreview\fi}
%    \end{macrocode}
%  \end{macro}
%
%    \begin{macrocode}
\DeclareVoidOption{beamer}{%
  \def\sa@class{beamer}%
  \sa@previewfalse
  \newenvironment{standalone}[1][]{%
  \begin{frame}[fragile,environment=standalone]}{\end{frame}}%
}
\DeclareStringOption[article]{class}
\DeclareStringOption[]{classoptions}
\DeclareStringOption[]{frameoptions}
\DeclareDefaultOption{%
  \xdef\sa@classoptionslist{\sa@classoptionslist,\CurrentOption}%
}
% vim: ft=tex
\PassOptionsToPackage{active}{preview}

\ProcessKeyvalOptions*\relax
%    \end{macrocode}
%
%    \begin{macrocode}
\let\@classoptionslist\sa@classoptionslist
\xdef\@tempa{[\sa@classoptions]{\sa@class}}
\expandafter\LoadClass\@tempa
%    \end{macrocode}
%
%    \begin{macrocode}
\ifsa@preview
  \RequirePackage{preview}
\fi
%    \end{macrocode}
%
%    \begin{macrocode}
\RequirePackage{standalone}[2010/03/20]
\standalonetrue
%    \end{macrocode}
%
%  \begin{macro}{\document}
%    \begin{macrocode}
\def\document{%
  \sa@orig@document
  \let\documentclass\sa@documentclass
  \sa@cls@document
}
%    \end{macrocode}
%  \end{macro}
%
%  \begin{macro}{\enddocument}
%    \begin{macrocode}
\def\enddocument{%
  \sa@cls@enddocument
  \sa@orig@enddocument
}
%    \end{macrocode}
%  \end{macro}
%
% \iffalse
%</cls>
% \fi
%
% \subsection{Simple TeX File}
% \iffalse
%<*tex>
% \fi
%    \begin{macrocode}
\expandafter\ifx\csname ifstandalone\endcsname\relax
%    \end{macrocode}
%    \begin{macrocode}
\expandafter\newif\csname ifstandalone\endcsname
%    \end{macrocode}
%    \begin{macrocode}
\expandafter\ifx\csname @twoclasseserror\endcsname\documentclass
\else
  \standalonetrue
\fi
%    \end{macrocode}
%    \begin{macrocode}
\fi
%    \end{macrocode}
% \iffalse
%</tex>
% \fi
%
% \subsection{Config File}
% \iffalse
%<*cfg>
% \fi
%    \begin{macrocode}
\PassOptionsToPackage{active,tightpage}{preview}%
%    \end{macrocode}
% \iffalse
%</cfg>
% \fi
%
% \iffalse
%<cfg>% vim: ft=tex
% \fi
% \Finale
% \endinput
 % use before any '\documentclass'
%   \ifstandalone
%     % Used only if compiled standalone
%   \fi
% \end{lstlisting}
%
% \subsection{Usage with \texttt{svn-multi} keywords}\label{sec:svnmulti}
% If the version control package \texttt{svn-multi} is used, the keyword macros (|\svnid| or |\svnidlong|) 
% need to be placed after the |\begin{document}| to be taken into account by the main document.
% The \texttt{svn-multi} package must be loaded by the sub-file preamble to avoid compile errors in standalone mode.
% Alternative, if the keywords are not required in this mode, they can be surrounded by |\ifstandalone\else|\dots|\fi|.
%
% \StopEventually{}
% \clearpage
%
% \section{Implementation}
%
% \subsection{The Package File}
% \iffalse
%<*sty>
% \fi
%    \begin{macrocode}
\expandafter\newif\csname ifstandalone\endcsname
\standalonefalse
%    \end{macrocode}
%    \begin{macrocode}
\@ifundefined{ifstandalonebeamer}{%
\expandafter\newif\csname ifstandalonebeamer\endcsname
\standalonebeamerfalse
}{}%
%    \end{macrocode}
%
% \begin{macro}{\sa@documentclass}
%    \begin{macrocode}
\newcommand{\sa@documentclass}[2][]{%
  \let\document\sa@document
  \expandafter\sa@@documentclass
}
%    \end{macrocode}
% \end{macro}
%
% \begin{macro}{\sa@@documentclass}
%    \begin{macrocode}
\def\sa@@documentclass{%
  \begingroup\def\sa@gobbleto{document}\sa@gobble
}
%    \end{macrocode}
% \end{macro}
%
% \begin{macro}{\sa@gobble}
%    \begin{macrocode}
\long\def\sa@gobble#1\begin#2{%
  \def\@tempa{#2}%
  \ifx\@tempa\sa@gobbleto
    \def\next{\expandafter\endgroup\expandafter\begin\expandafter{\sa@gobbleto}}%
  \else
    \def\next{\sa@gobble}%
  \fi
  \next
}
%    \end{macrocode}
% \end{macro}
%
%    \begin{macrocode}
\@ifundefined{standalone}
  {\newenvironment{standalone}[1][]{}{}}
  {}
%    \end{macrocode}
%
% Gobbles all arguments: |<...>[<...>][...]{...}{...}|. Please note that the last two |{ }| arguments are also optional.
%    \begin{macrocode}
\@ifundefined{standaloneframe}
  {\@ifundefined{beamer@newenv}
    {\newenvironment{standaloneframe}[1][]{%
      \@ifnextchar[%]
        {\sa@framegobbleopt}{\sa@framegobbleargs}}{}%
    }
    {\newenvironment<>{standaloneframe}[1][]{%
      \@ifnextchar[%]
        {\sa@framegobbleopt}{\sa@framegobbleargs}}{}%
    }
   \def\sa@framegobbleopt[#1]{\sa@framegobbleargs}
   \def\sa@framegobbleargs{%
     \@ifnextchar\bgroup
       {\sa@framegobbleargs@}%
       {}%
   }
   \def\sa@framegobbleargs@#1{%
    \@ifnextchar\bgroup
      {\@gobble}%
      {}%
   }
  }
  {}
%    \end{macrocode}
%
% \begin{macro}{\sa@orig@document}
%    \begin{macrocode}
\let\sa@orig@document\document
% \end{macro}
%
% \begin{macro}{\sa@orig@enddocument}
%    \begin{macrocode}
\let\sa@orig@enddocument\enddocument
%    \end{macrocode}
% \end{macro}
%
% \begin{macro}{\document}
%    \begin{macrocode}
\def\document{%
  \sa@orig@document
  \let\documentclass\sa@documentclass
  \ignorespaces
}
%    \end{macrocode}
% \end{macro}
%
% \begin{macro}{\sa@document}
%    \begin{macrocode}
\def\sa@document{%
  \let\enddocument\sa@enddocument
  \sa@atbegindocument
}
%    \end{macrocode}
% \end{macro}
%
% \begin{macro}{\sa@enddocument}
%    \begin{macrocode}
\def\sa@enddocument{%
  \sa@atenddocument
  \aftergroup\sa@@enddocument
}
%    \end{macrocode}
% \end{macro}
%
% \begin{macro}{\sa@atbegindocument}
%    \begin{macrocode}
\def\sa@atbegindocument{%
  \ignorespaces
}%
%    \end{macrocode}
% \end{macro}
%
% \begin{macro}{\sa@atenddocument}
%    \begin{macrocode}
\def\sa@atenddocument{%
  \unskip
}%
%    \end{macrocode}
% \end{macro}
%
% \begin{macro}{\sa@@enddocument}
%    \begin{macrocode}
\def\sa@@enddocument{%
  %\let\document\sa@orig@document
  \let\enddocument\sa@orig@enddocument
  \endinput
}
%    \end{macrocode}
% \end{macro}
%
% \begin{macro}{\sa@processpreamble}
%    \begin{macrocode}
\def\sa@processpreamble{%
  \renewcommand\usepackage[2][]{%
    \message{^^J%
      INFO: Sub-file requires the following package(s):^^J
      \space\space[##1]{##2}^^J%
    }%
  }%
  \let\RequirePackage\usepackage
}
%    \end{macrocode}
% \end{macro}
%
%    \begin{macrocode}
%%\def\sa@@documentclass{\sa@processpreamble}
%    \end{macrocode}
%
% \iffalse
%</sty>
% \fi
%
% \subsection{The Class File}
% \iffalse
%<*cls>
% \fi
%    \begin{macrocode}
\def\sa@classoptionslist{}
\RequirePackage{kvoptions}
\SetupKeyvalOptions{prefix=sa@}
\DeclareBoolOption[true]{preview}
%    \end{macrocode}
%
% \begin{environment}{standalone}
% The \env{standalone} environment is defined by default to be without effect.
% The \cs{endstandalone} macro is set to |\relax|, so a redefinition with
% \cs{renewenvironment} can be detected later.
%    \begin{macrocode}
\let\standalone\empty
\let\endstandalone\relax
%    \end{macrocode}
% \end{environment}
%
% \begin{macro}{\sa@cls@document}
% \begin{macro}{\sa@cls@enddocument}
%    \begin{macrocode}
\def\sa@cls@document{\standalone}
\def\sa@cls@enddocument{\endstandalone}
%    \end{macrocode}
% \end{macro}
% \end{macro}
%
% The |beamer| option defines the \env{standalone} environment as a replacement
% of \env{frame}.
%    \begin{macrocode}
\newif\ifstandalonebeamer
\DeclareVoidOption{beamer}{%
  \def\sa@class{beamer}%
  \sa@previewfalse
  \standalonebeamertrue
  \newenvironment{standaloneframe}{%
    \@ifnextchar<%
      {\@standaloneframe}%
      {\@@standaloneframe{}}%
  }{\end{frame}}%
  \def\@standaloneframe<##1>{%
      \@@standaloneframe{<##1>}%
  }
  \def\@@standaloneframe##1{%
    \@ifnextchar[%]
      {\@@@standaloneframe{##1}}%
      {\@@@standaloneframe{##1}[]}%
  }%
  \def\@@@standaloneframe##1[{%
    \@ifnextchar<%
      {\@@@@standaloneframe{##1}[}%
      {\@@@@@@standaloneframe{##1}[}%
  }%
  \def\@@@@standaloneframe##1[##2]{%
    \@ifnextchar[%]
      {\@@@@@standaloneframe{##1}{##2}}%
      {\begin{frame}##1[##2][environment=standaloneframe]}%
  }%
  \def\@@@@@standaloneframe##1##2[##3]{%
      \begin{frame}##1[##2][environment=standaloneframe,##3]%
  }%
  \def\@@@@@@standaloneframe##1[##2]{%
      \begin{frame}##1[environment=standaloneframe,##2]%
  }%
}
%    \end{macrocode}
%
%    \begin{macrocode}
\DeclareStringOption[article]{class}
\DeclareDefaultOption{%
  \xdef\sa@classoptionslist{\sa@classoptionslist,\CurrentOption}%
}
\ProcessKeyvalOptions*\relax
%    \end{macrocode}
%
%    \begin{macrocode}
\let\@classoptionslist\sa@classoptionslist
\xdef\@tempa{[\sa@classoptionslist]{\sa@class}}
\expandafter\LoadClass\@tempa
%    \end{macrocode}
%
%    \begin{macrocode}
% vim: ft=tex
\PassOptionsToPackage{active}{preview}

%    \end{macrocode}
%
% \subsubsection*{Preview Code}
% The \env{standalone} environment is redefined to use the |preview| environment as
% long it was not redefined in the configuration file.
%    \begin{macrocode}
\ifsa@preview
  \@ifundefined{endstandalone}{%
    \renewenvironment{standalone}
      {\preview   }
      {\endpreview}
  }{}%
  \RequirePackage{preview}
\fi
%    \end{macrocode}
%
%    \begin{macrocode}
\RequirePackage{standalone}[2010/03/21]
\standalonetrue
%    \end{macrocode}
%
% \begin{macro}{\document}
%    \begin{macrocode}
\def\document{%
  \sa@orig@document
  \let\documentclass\sa@documentclass
  \sa@cls@document
}
%    \end{macrocode}
% \end{macro}
%
% \begin{macro}{\enddocument}
%    \begin{macrocode}
\def\enddocument{%
  \sa@cls@enddocument
  \sa@orig@enddocument
}
%    \end{macrocode}
% \end{macro}
%
% \iffalse
%</cls>
% \fi
%
% \subsection{Simple TeX File}
% \iffalse
%<*tex>
% \fi
%    \begin{macrocode}
\expandafter\ifx\csname ifstandalone\endcsname\relax
%    \end{macrocode}
%    \begin{macrocode}
\expandafter\newif\csname ifstandalone\endcsname
%    \end{macrocode}
%    \begin{macrocode}
\expandafter\ifx\csname @twoclasseserror\endcsname\documentclass
\else
  \standalonetrue
\fi
%    \end{macrocode}
%    \begin{macrocode}
\fi
%    \end{macrocode}
% \iffalse
%</tex>
% \fi
%
% \subsection{Config File}
% Default content of the configuration file. Users can place there own
% |standalone.cfg| in their |texmf| directory or in the local document directory
% to define their own settings as described above. To load the default config file
% from a user config file use |% vim: ft=tex
\PassOptionsToPackage{active}{preview}
|.
% \iffalse
%<*cfg>
% \fi
%    \begin{macrocode}
\PassOptionsToPackage{active,tightpage}{preview}%

\ifstandalonebeamer
  \setbeamertemplate{navigation symbols}{}
\fi
%    \end{macrocode}
% \iffalse
%</cfg>
% \fi
%
% \iffalse
%<cfg>% vim: ft=tex
% \fi
% \Finale
% \endinput
