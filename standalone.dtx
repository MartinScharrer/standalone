% \iffalse meta-comment
%<=*COPYRIGHT>
%% Copyright (C) 2011 by Martin Scharrer <martin@scharrer-online.de>
%% ---------------------------------------------------------------------------
%% This work may be distributed and/or modified under the
%% conditions of the LaTeX Project Public License, either version 1.3
%% of this license or (at your option) any later version.
%% The latest version of this license is in
%%   http://www.latex-project.org/lppl.txt
%% and version 1.3 or later is part of all distributions of LaTeX
%% version 2005/12/01 or later.
%%
%% This work has the LPPL maintenance status `maintained'.
%%
%% The Current Maintainer of this work is Martin Scharrer.
%%
%% This work consists of the files <+name+>.dtx and <+name+>.ins
%% and the derived filebase <+name+>.sty.
%%
%<=/COPYRIGHT>
% \fi
%
% \iffalse
%<*driver>
\NeedsTeXFormat{LaTeX2e}
\ProvidesFile{standalone.dtx}[%
%<=*DATE>
    2011/11/01
%<=/DATE>
%<=*VERSION>
    v0.5
%<=/VERSION>
    DTX file of the standalone bundle]
\documentclass[a4paper,11pt]{ydoc}
%\addtolength{\oddsidemargin}{1cm}
%\addtolength{\textwidth}{-1cm}

\EnableCrossrefs
\CodelineIndex
\RecordChanges
%\OnlyDescription
\begin{document}
  \DocInput{\jobname.dtx}
  \PrintChanges
  \newpage\PrintIndex
\end{document}
%</driver>
% \fi
%
% \CheckSum{0}
%
% \CharacterTable
%  {Upper-case    \A\B\C\D\E\F\G\H\I\J\K\L\M\N\O\P\Q\R\S\T\U\V\W\X\Y\Z
%   Lower-case    \a\b\c\d\e\f\g\h\i\j\k\l\m\n\o\p\q\r\s\t\u\v\w\x\y\z
%   Digits        \0\1\2\3\4\5\6\7\8\9
%   Exclamation   \!     Double quote  \"     Hash (number) \#
%   Dollar        \$     Percent       \%     Ampersand     \&
%   Acute accent  \'     Left paren    \(     Right paren   \)
%   Asterisk      \*     Plus          \+     Comma         \,
%   Minus         \-     Point         \.     Solidus       \/
%   Colon         \:     Semicolon     \;     Less than     \<
%   Equals        \=     Greater than  \>     Question mark \?
%   Commercial at \@     Left bracket  \[     Backslash     \\
%   Right bracket \]     Circumflex    \^     Underscore    \_
%   Grave accent  \`     Left brace    \{     Vertical bar  \|
%   Right brace   \}     Tilde         \~}
%
% \changes{v0.1}{2010/03/21}{First released version}
% \changes{v0.2}{2010/03/22}{Added support for beamer style settings in the config file using \cs{ifstandalonebeamer}. Updated documentation.}
% \changes{v0.3}{2010/03/26}{Added package options to automatically copy sub-preamble code to the main preamble.}
% \changes{v0.3a}{2010/03/27}{Fixed bug related to line endings in \texttt{.sta} file. Listed sub-files now have \texttt{.tex} as default extension.}
% \changes{v0.4}{2011/02/28}{Added option 'border' for easy adjustments of the preview border.}
%
% \GetFileInfo{\jobname.dtx}
%
% \DoNotIndex{\newcommand,\newenvironment,\def,\edef,\xdef,\gdef,\DeclareRobustCommand}
% \DoNotIndex{\expandafter,\if,\else,\fi,\ifnum,\ifx,\let,\global,\long}
% \DoNotIndex{\newcounter,\newcount,\message,\meaning,\noexpand,\relax,\value}
% \DoNotIndex{\setcounter,\addtocounter,\advance,\afterassignment,\AtEndOfPackage}
% \DoNotIndex{\ProvidesPackage,\providecommand,\RequirePackage,\empty,\begin,\end}
% \DoNotIndex{\begingroup,\bgroup,\egroup,\endgroup,\csname,\endcsname,\@tempa,\@tempb}
% \DoNotIndex{\ignorespaces,\lccode,\sffamily,\@gobble,\@ifundefined,\@for,\or}
% \DoNotIndex{\@firstoftwo,\@ifnextchar,\@namedef,\@nameuse,\@secondoftwo}
% \DoNotIndex{\@temptokena,\toks@,\BODY,\do,\g@addto@macro,\lowercase,\uppercase,\the}
% \DoNotIndex{\aftergroup,\CurrentOption,\DeclareBoolOption,\DeclareDefaultOption,\DeclareStringOption,\DeclareVoidOption}
% \DoNotIndex{\endinput,\endpreview,\input,\LoadClass,\newif,\PassOptionsToPackage,\preview,\ProcessKeyvalOptions}
% \DoNotIndex{\renewcommand,\renewenvironment,\setbeamertemplate,\SetupKeyvalOptions,\space,\unskip,\usepackage}
% \DoNotIndex{\unexpanded,\write,\next,\jobname,\hbox}
%
% \ifpdf
% \hypersetup{%
%   pdfauthor   = {Martin Scharrer <martin@scharrer-online.de>},
%   pdftitle    = {The standalone class and package},
%   pdfsubject  = {Documentation of LaTeX class and package 'standalone'},
%   pdfkeywords = {standalone, LaTeX, TeX}
% }%
% \fi
% \clearpage
% \null
% \vspace*{-2em}
% \begin{center}
%   {\LARGE\textsf{The \textbf{standalone} Class and Package\\[\bigskipamount]}}
%   {\large Martin Scharrer \\[\smallskipamount]\normalsize
%   \url{martin@scharrer-online.de}\\[\medskipamount]
%   \url{http://www.ctan.org/pkg/standalone/}\\[\bigskipamount]}
%   {\large Version \fileversion\ -- \filedate}\\
% \end{center}
% \vspace{1.2em}%
%
% \begin{abstract}
% The \pkg{standalone} class and package allow the standalone compilation of external \LaTeX\ files holding
% picture or similar environments (or any other material) while still allow to |\input| this files directly
% to a main document. Advances features include the automatic transfer of the preambles of the external
% files to the main preamble, the conversion of the standalone files into images or the automatically
% compilation of included standalone files from the main document.
% \end{abstract}
%
% \begin{multicols}{2}
% \setcounter{tocdepth}{2}
% \tableofcontents
% \end{multicols}
%
% \section{Introduction}
% Larger \LaTeX{} documents can be split into multiple \TeX\ files which are then included in a main document with \cs{include}
% for e.g.\ chapter files or \cs{input} for e.g.\ \TeX-coded pictures. Keeping pictures in their own sub-files improves readability
% of the main file and simplifies the sharing of them between different documents. However, during the, sometimes lengthly,
% drawing/coding process it has benefits to be able to compile the pictures on their own. The compile process is much quicker and
% the resulting document only holds the picture which avoids constant page turning and zooming.
%
% While it is possible to write a small `main' file for each picture file, this method is a little cumbersome and clutters
% the directories with a lot of extra files. A second method is to place the `main' components, i.e. a preamble, directly into the
% picture files and make the main document ignore this code sections.
%
% The package \cls{standalone} can be used in the main document to skip all extra preambles in included files. The main file must load all packages
% and settings required by the sub-files. Several package options are provided to collect the preambles of the sub-files automatically and execute them from the main file.
%
% A \cls{standalone} class is also provided to minimise the extra preamble code needed in this files.
% It's usage is optional, but simplifies and standardises how picture files are compiled standalone.
% The class uses by default the \pkg{preview} package to create an output file which only contains the picture with no extra
% margins, page numbers or anything else. A configuration file \file{standalone.cfg} read by the class allows the user to adjust settings and macros easily
% on a per directory base.
%
% \subsection{Similar Packages}
% The \pkg{docmute} package by T.M.\ Trzeciak is written for the same basic task as the \cls{standalone} package. However, no sub-preamble processing other than the removal is support. It also
% doesn't provide a special class or configuration file.
%
% The \lib{external} library of \pkg{tikz} allows to externalize \env{tikzpictures} from an main document. Its build feature is similar to the one provided by \cls{standalone}.
% However, both work form different directions: \cls{standalone} allows to include external \env{tikzpictures} to be included in a main file while ignoring the preamble while
% \lib{external} writes them from the main file to temporary external files. The user must decide which workflow is better suited for him/her.
% Also \cls{standalone} is working independently of \pkg{tikz} and supports other picture environments like \pkg{pstricks} or any other \TeX\ material.
%
% \subsection{Dependencies}
% The \cls{standalone} class and package require the \pkg{pgfkeys} package (PGF/TikZ bundle) as well as the \pkg{svn-prov} package.
% The \pkg{preview} package is required by the \opt{preview} option.
% The \pkg{beamer} option of course requires the \pkg{beamer} bundle to be installed.
%
% The \pkg{standalone} package requires the \pkg{currfile} package (which in turn uses \pkg{filehook}) to track the correct file names of
% sub-files included using |\input|.
% For the compilation support for included standalone files the \pkg{gincltex} and \pkg{filemod} packages are also required.
%
% To compile the documentation of \cls{standalone} the \cls{ydoc} bundle is required.
%
% All of these packages are included in recent versions of the TeXLive or MikTeX distributions and are freely available on CTAN (\url{http://www.ctan.org/}).
%
%
% \section{Usage}
% \subsection{Quick instructions}
% Load the \cls{standalone} \emph{package} very early in the main document. Also all packages needed by all the sub-files must be loaded by the main document.
% Include your picture or other sub-files using |\input| or a similar macro as normal.
% In the sub-files use the \cls{standalone} \emph{class} with a normal \cs{documentclass} and load all packages needed for the particular
% file. Finally wrap the actual content of the sub-file in a |document| environment.
%
% When the sub-file is compiled on its own the |\documentclass| and |document| environment will be active as normal. The main file, however,
% will skip everything from the |\documentclass| till the |\begin{document}|. The (now fake) |document| environment is redefined to be a simple
% TeX-group. Any code after the |\end{document}| will be ignored. The real |document| environment of the main file will be unaffected and will work as normal.
%
% The packages required by each sub-file can be transfered automatically to the main document preamble using the options
% listed in section~\ref{sec:pkgoptions}.
%
% \subsection{Class Options}
% The \cls{standalone} class will load a real document class.\optpar{class} By default this is |article|. The document class normally has not much
% influence on sub-files like pictures, especially when the \pkg{preview} package is active. However, the used class can be adjusted by the user
% with the \opt{class}|=|\meta{class name} option.
%
% The boolean \opt*{preview} option can be used to disable the use of the \pkg{review} package. The default is |preview=true|. The package is not loaded if |preview=false| is set.
% The preview border can be set using the \opt*{border} option. This border will be added to the file content.
% It can be used in one of the three following ways:
% \begin{quote}%
%  \MacroArgs!\opt{\texttt{border}}!'='<all sides> \\
%  \MacroArgs!\opt{\texttt{border}}!'='{<left>~<lower>~<right>~<upper>} \\
%  \MacroArgs!\opt{\texttt{border}}!'='{<left/right>~<lower/upper>}
% \end{quote}
% The default unit is |bp| (big points) which is the unit used by the PostScript and PDF formats. If |preview=false| is set this option will be ignored.
%
% A special \opt*{beamer} option is provided to handle beamer frames and overlays correctly. See section~\ref{sec:beamer} for more information.
%
% The \opt*{convert} option can be used to convert the output file to a image. See the section~\ref{sec:conv} for more information.
%
% All other used options are passed to the loaded class.
%
% \subsection{Package Options}\label{sec:pkgoptions}
% The \cls{standalone} package removes all sub-file preambles (``sub-preambles'') by default when loaded.
% However, if the package is loaded with the \opt*{subpreambles} options, all sub-preambles are stored in an auxiliary file with the name `\meta{main tex file name}|.sta|' (for \emph{sta}ndalone).
% This file is then loaded or processed at the beginning of the next \LaTeX\ run (i.e.\ at the place in the preamble where the \cls{standalone} package is loaded).
% The way how the \opt{subpreambles} option works can be controlled by the options \opt{sort}, \opt{print} and \opt{comments}/\opt{nocomments}.
% Please note that the \opt{sort} and \opt{print} options require of course the \opt{subpreambles} option and will enable it if not already done so.
%
% With only the \opt{subpreambles} option set, the sub-preambles are simple read and executed unchanged. This includes the risk of option clashes if one package is loaded with different
% options inside the sub-preambles and/or the main preamble. This is avoided by the \opt*{sort} option, which accumulates all packages loaded by all sub-files together with their options.
% The options are then marked to be loaded by the package using \LaTeX{}s \cs{PassOptionsToPackage} macro. The packages are loaded at the end of the preamble using the \cs{AtBeginDocument} hook.
% This allows the user to load the same packages with own options in the main file, after the |subversion| package is loaded, without any option clashes.
%
% While the \opt{sort} option is giving already good results, problems with the order of packages can still occur. Some packages provide, redefine or patch the same macros, so that they must be loaded in
% the correct order to give the desired result. Potential additional code in the sub-preambles, required for some sub-figures but maybe incompatible with others, complicates the situation further.
% If such issues occur they can hardly be handled in an automatic way. Instead the sub-preambles must be carefully merged into the main preamble.
% The option \opt*{print} was created to simplify this otherwise cumbersome task.
% It concatenated all sub-preambles into a single file named `\meta{main tex file name}|.stp|' (for \emph{st}andalone, \emph{p}rint). Each preamble is commented with its original file name.
% Please note that |.sta| file mentioned above, while quite similar, holds additional macros and might not be easily user readable or editable.
% After the file was generated it can be easily pasted into the main file preamble using a text editor.
%
% When the \opt{print} option is enabled the normal |.sta| file is not generated or loaded. Because this will cause most likely some errors related to packages not loaded, all sub-file bodies will be skipped.
% A warning is printed for each sub-file to remind the user about this fact. The \opt{print} option is only indented to by used when required to get a list of sub-preambles. After including this list in the main file the option
% must be removed to compile the main file normally.
%
% Finally if both the \opt*{print} and \opt*{sort} options are enabled, a `sorted' list of sub-preambles is printed into the |.stp| file. In this `sorted print' mode all |\usepackage| macros (and similar macros like
% |\usepgflibrary|, |\usetikzlibrary| as well as |\usetikztiminglibrary| from the |pgf|, |tikz| and |tikz-timing| packages, respectively) are removed from the rest of the sub-preamble code.
% A list of packages (and libraries) without duplicates is printed at the begin of the |.stp| file (using |\usepackage|, of course). Every option provided by any sub-file for a package is added, again without duplicates.
% If specific package date was requested in a sub-file it is also added. If multiple dates are requested for one package, the most recent (i.e.\ the ``highest one'', not the last processed) is used.
% After this list(s) the rest of the sub-preamble code is printed with the above macros removed. This mode frees the user from the need to remove duplicates and collect package options manually.
%
% Please note that all |\usepackage| and similar macros inside braces |{}| will not be seen by \cls{standalone}s sort macro and therefore are not extracted or handled in any special form mentioned above.
% This can be exploited to load certain packages only in \cls{standalone} mode but not in the main document.
% Unfortunately, macros inside |\ifstandalone|\ldots|\fi| are seen and extracted while not wanted inside the main file.
% The macro \cs{onlyifstandalone}\marg{code} (see below) was created because of this two reasons. Its argument braces hide the content from the scanner.
% It is then also completely removed from the printed sub-preamble code.
%
% The complementary options \opt*{comments}/\opt*{nocomments} select if the |.stp| file should also include the comments of the sub-preambles.
% Comments are included by default in the non-sorting print mode (\opt{print} without \opt{sort} option), but can cause `wrong' results during the `sorting' process and are therefore removed by default in this mode.
% The reason for this can be explained as follows.
% In order to transfer the comments from the sub-files to the |.stp| file \TeX\ must be instructed to handle them as normal input and not discard them.
% However, in this case the scanning algorithm which removes |\usepackage| and friends can not distinguish
% between `active' macros and macros which are commented out. All above mentioned macro inside comments will then be processed as when there where `active'.
% The user might favour the information provided by the comments over this small risk and enable them using the \opt{comments} option.
%
% \subsection{Macros and Environments}
% No macros or environments need to be used to make \cls{standalone} work. Simply loading the class or package is enough for the basic usage.
% However the following macros and environments can be useful for changing the behaviour of \cls{standalone}.
%
% \DescribeMacro{\ifstandalone}
% Both the class and the package provide the if-switch \cs{ifstandalone}, which can be used to only include code if the file is
% compiled standalone. The switch is set to \cs{iftrue} by the class and to \cs{iffalse} by the package.
%
% The additional file \file{standalone.tex} also defines this switch by checking if \cs{documentclass} was already used. It can be included
% with |% \iffalse meta-comment
%
% Copyright (c) 2010 by Martin Scharrer <martin@scharrer-online.de>
% -----------------------------------------------------------------
%
% This work may be distributed and/or modified under the
% conditions of the LaTeX Project Public License, either version 1.3c
% of this license or (at your option) any later version.
% The latest version of this license is in
%
%   http://www.latex-project.org/lppl.txt
%
% and version 1.3c or later is part of all distributions of LaTeX
% version 2008/05/04 or later.
%
% This work has the LPPL maintenance status `maintained'.
%
% The Current Maintainer of this work is Martin Scharrer.
%
% This work consists of the files standalone.dtx, standalone.ins
% and the derived file standalone.sty.
%
% $Id$
% \fi
%
% \iffalse
%<cls|sty|cfg|tex>% $Id$
%<cls|sty>\NeedsTeXFormat{LaTeX2e}
%<cls>\ProvidesClass{standalone}   [2010/03/20 v0.1 Class to compile TeX sub-files standalone]
%<sty>\ProvidesPackage{standalone} [2010/03/20 v0.1 Package to include TeX sub-files which use the 'standalone' class]
%<tex>\ProvidesFile{standalone.tex}[2010/03/20 v0.1 Provides if-switch to show if file is compiled standalone]%
%<cfg>\ProvidesFile{standalone.cfg}[2010/03/20 v0.1 Default configuration file for 'standalone' class and package]%
%<*driver>
\NeedsTeXFormat{LaTeX2e}
\ProvidesFile{standalone.dtx}[2010/03/20 v0.1 DTX File for 'standalone' cls/sty/tex]
\documentclass{ltxdoc}
\usepackage{ifpdf}
\usepackage{hyperref}
\usepackage{tabularx}
\usepackage{array}
\usepackage{flafter,fnpos}
\usepackage{booktabs}
\usepackage{amsmath}
\usepackage{placeins}
\makeFNbottom
\makeFNbelow

\EnableCrossrefs
%\CodelineIndex
\RecordChanges
%\OnlyDescription
\begin{document}
  \DocInput{\jobname.dtx}
  \PrintChanges
  %\newpage\PrintIndex
\end{document}
%</driver>
% \fi
%
% \CheckSum{0}
%
% \CharacterTable
%  {Upper-case    \A\B\C\D\E\F\G\H\I\J\K\L\M\N\O\P\Q\R\S\T\U\V\W\X\Y\Z
%   Lower-case    \a\b\c\d\e\f\g\h\i\j\k\l\m\n\o\p\q\r\s\t\u\v\w\x\y\z
%   Digits        \0\1\2\3\4\5\6\7\8\9
%   Exclamation   \!     Double quote  \"     Hash (number) \#
%   Dollar        \$     Percent       \%     Ampersand     \&
%   Acute accent  \'     Left paren    \(     Right paren   \)
%   Asterisk      \*     Plus          \+     Comma         \,
%   Minus         \-     Point         \.     Solidus       \/
%   Colon         \:     Semicolon     \;     Less than     \<
%   Equals        \=     Greater than  \>     Question mark \?
%   Commercial at \@     Left bracket  \[     Backslash     \\
%   Right bracket \]     Circumflex    \^     Underscore    \_
%   Grave accent  \`     Left brace    \{     Vertical bar  \|
%   Right brace   \}     Tilde         \~}
%
%
% \changes{v0.1}{2010/03/20}{First released version}
%
% \GetFileInfo{\jobname.dtx}
%
% \DoNotIndex{\newcommand,\newenvironment,\def,\edef,\xdef,\DeclareRobustCommand}
% \DoNotIndex{\expandafter,\if,\else,\fi,\ifnum,\ifx,\let,\global,\long}
% \DoNotIndex{\newcounter,\newcount,\message,\meaning,\noexpand,\relax,\value}
% \DoNotIndex{\setcounter,\addtocounter,\advance,\afterassignment,\AtEndOfPackage}
% \DoNotIndex{\ProvidesPackage,\providecommand,\RequirePackage,\empty,\begin,\end}
% \DoNotIndex{\begingroup,\bgroup,\egroup,\endgroup,\csname,\endcsname,\@tempa,\@tempb}
% \DoNotIndex{\ignorespaces,\lccode,\sffamily,\@gobble,\@ifundefined,\@for,\or}
% \DoNotIndex{\@firstoftwo,\@ifnextchar,\@namedef,\@nameuse,\@secondoftwo}
% \DoNotIndex{\@temptokena,\toks@,\BODY,\do,\g@addto@macro,\lowercase,\uppercase,\the}
%
% \ifpdf
% \hypersetup{%
%   pdfauthor   = {Martin Scharrer <martin@scharrer-online.de>},
%   pdftitle    = {The standalone class and package},
%   pdfsubject  = {Documentation of LaTeX class and package 'standalone'},
%   pdfkeywords = {standalone, LaTeX, TeX}
% }%
% \fi
% \clearpage
% \null
% \vspace*{-2em}
% \begin{center}
%   {\LARGE\sffamily The \emph{standalone} Class and Package\\}
%   {\large Martin Scharrer \\\normalsize 
%   \url{martin@scharrer-online.de}\\[.8ex]
%   \url{http://www.ctan.org/pkg/standalone/}\\[1.2ex]}
%   {\large Version \fileversion\ -- \filedate}\\
% \end{center}
% \vspace{1.2em}%
%
% \section{Introduction}
%
% \StopEventually{}
% \FloatBarrier
% \clearpage
%
% \section{Implementation}
%
% \subsection{The Package File}
% \iffalse
%<*sty>
% \fi
%    \begin{macrocode}
\expandafter\newif\csname ifstandalone\endcsname
\standalonefalse
%    \end{macrocode}
%
%  \begin{macro}{\sa@documentclass}
%    \begin{macrocode}
\newcommand{\sa@documentclass}[2][]{%
  \let\document\sa@document
  \let\enddocument\sa@enddocument
  \expandafter\sa@@documentclass
}
%    \end{macrocode}
%  \end{macro}
%
%  \begin{macro}{\sa@@documentclass}
%    \begin{macrocode}
\def\sa@@documentclass{%
  \begingroup\def\sa@gobbleto{document}\sa@gobble
}
%    \end{macrocode}
%  \end{macro}
%
%  \begin{macro}{\sa@gobble}
%    \begin{macrocode}
\long\def\sa@gobble#1\begin#2{%
  \def\@tempa{#2}%
  \ifx\@tempa\sa@gobbleto
    \def\next{\expandafter\endgroup\expandafter\begin\expandafter{\sa@gobbleto}}%
  \else
    \def\next{\sa@gobble}%
  \fi
  \next
}
%    \end{macrocode}
%  \end{macro}
%
%    \begin{macrocode}
\expandafter\ifx\csname standalone\endcsname\relax
  \newenvironment{standalone}[1][]{}{}
\fi
%    \end{macrocode}
%
%  \begin{macro}{\sa@orig@document}
%    \begin{macrocode}
\let\sa@orig@document\document
%  \end{macro}
%
%  \begin{macro}{\sa@orig@enddocument}
%    \begin{macrocode}
\let\sa@orig@enddocument\enddocument
%    \end{macrocode}
%  \end{macro}
%
%  \begin{macro}{\document}
%    \begin{macrocode}
\def\document{%
  \sa@orig@document
  \let\documentclass\sa@documentclass
  \ignorespaces
}
%    \end{macrocode}
%  \end{macro}
%
%  \begin{macro}{\sa@document}
%    \begin{macrocode}
\def\sa@document{%
  \sa@atbegindocument
}
%    \end{macrocode}
%  \end{macro}
%
%  \begin{macro}{\sa@enddocument}
%    \begin{macrocode}
\def\sa@enddocument{%
  \sa@atenddocument
  \aftergroup\sa@@enddocument
}
%    \end{macrocode}
%  \end{macro}
%
%  \begin{macro}{\sa@atbegindocument}
%    \begin{macrocode}
\def\sa@atbegindocument{%
  \ignorespaces
}%
%    \end{macrocode}
%  \end{macro}
%
%  \begin{macro}{\sa@atenddocument}
%    \begin{macrocode}
\def\sa@atenddocument{%
  \unskip
}%
%    \end{macrocode}
%  \end{macro}
%
%  \begin{macro}{\sa@@enddocument}
%    \begin{macrocode}
\def\sa@@enddocument{%
  %\let\document\sa@orig@document
  \let\enddocument\sa@orig@enddocument
  \endinput
}
%    \end{macrocode}
%  \end{macro}
%
%  \begin{macro}{\sa@processpreamble}
%    \begin{macrocode}
\def\sa@processpreamble{%
  \renewcommand\usepackage[2][]{%
    \message{^^J%
      INFO: Sub-file requires the following package(s):^^J
      \space\space[##1]{##2}^^J%
    }%
  }%
  \let\RequirePackage\usepackage
}
%    \end{macrocode}
%  \end{macro}
%
%    \begin{macrocode}
%%\def\sa@@documentclass{\sa@processpreamble}
%    \end{macrocode}
%
% \iffalse
%</sty>
% \fi
%
% \subsection{The Class File}
% \iffalse
%<*cls>
% \fi
%    \begin{macrocode}
\def\sa@classoptionslist{}
\RequirePackage{kvoptions}
\SetupKeyvalOptions{prefix=sa@}
\DeclareBoolOption[true]{preview}
%    \end{macrocode}
%
%  \begin{macro}{\sa@cls@document}
%    \begin{macrocode}
\def\sa@cls@document{\ifsa@preview\preview\fi}
%    \end{macrocode}
%  \end{macro}
%
%  \begin{macro}{\sa@cls@enddocument}
%    \begin{macrocode}
\def\sa@cls@enddocument{\ifsa@preview\endpreview\fi}
%    \end{macrocode}
%  \end{macro}
%
%    \begin{macrocode}
\DeclareVoidOption{beamer}{%
  \def\sa@class{beamer}%
  \sa@previewfalse
  \newenvironment{standalone}[1][]{%
  \begin{frame}[fragile,environment=standalone]}{\end{frame}}%
}
\DeclareStringOption[article]{class}
\DeclareStringOption[]{classoptions}
\DeclareStringOption[]{frameoptions}
\DeclareDefaultOption{%
  \xdef\sa@classoptionslist{\sa@classoptionslist,\CurrentOption}%
}
% vim: ft=tex
\PassOptionsToPackage{active}{preview}

\ProcessKeyvalOptions*\relax
%    \end{macrocode}
%
%    \begin{macrocode}
\let\@classoptionslist\sa@classoptionslist
\xdef\@tempa{[\sa@classoptions]{\sa@class}}
\expandafter\LoadClass\@tempa
%    \end{macrocode}
%
%    \begin{macrocode}
\ifsa@preview
  \RequirePackage{preview}
\fi
%    \end{macrocode}
%
%    \begin{macrocode}
\RequirePackage{standalone}[2010/03/20]
\standalonetrue
%    \end{macrocode}
%
%  \begin{macro}{\document}
%    \begin{macrocode}
\def\document{%
  \sa@orig@document
  \let\documentclass\sa@documentclass
  \sa@cls@document
}
%    \end{macrocode}
%  \end{macro}
%
%  \begin{macro}{\enddocument}
%    \begin{macrocode}
\def\enddocument{%
  \sa@cls@enddocument
  \sa@orig@enddocument
}
%    \end{macrocode}
%  \end{macro}
%
% \iffalse
%</cls>
% \fi
%
% \subsection{Simple TeX File}
% \iffalse
%<*tex>
% \fi
%    \begin{macrocode}
\expandafter\ifx\csname ifstandalone\endcsname\relax
%    \end{macrocode}
%    \begin{macrocode}
\expandafter\newif\csname ifstandalone\endcsname
%    \end{macrocode}
%    \begin{macrocode}
\expandafter\ifx\csname @twoclasseserror\endcsname\documentclass
\else
  \standalonetrue
\fi
%    \end{macrocode}
%    \begin{macrocode}
\fi
%    \end{macrocode}
% \iffalse
%</tex>
% \fi
%
% \subsection{Config File}
% \iffalse
%<*cfg>
% \fi
%    \begin{macrocode}
\PassOptionsToPackage{active,tightpage}{preview}%
%    \end{macrocode}
% \iffalse
%</cfg>
% \fi
%
% \iffalse
%<cfg>% vim: ft=tex
% \fi
% \Finale
% \endinput
| and is intended for specialised files which do not use the \cls{standalone} class.
%
% \DescribeMacro{\onlyifstandalone}{<code>}
% This is the macro version of the |\ifstandalone| if-switch. It executes \meta{code} only in \cls{standalone} mode. As mentioned in section~\ref{sec:pkgoptions}
% it can also be used to hide |\usepackage| and similar macros from the extraction scanner of the |sort| option. The macro and its argument is not printed into the |.stp| file.
%
% \DescribeMacro{\standaloneignore}
% In rare cases some code must be placed before the |\documentclass| of a sub-file (e.g.\ |\PassOptionsToPackage|).
% Because the main document will only skip code between |\documentclass| and |\begin{document}| this code will be executed by it.
% In order to avoid this the macro \cs{standaloneignore} can be used at the very beginning of a sub-file to skip over this code.
% However it must be written as |\csname standaloneignore\endcsname| to avoid a `Undefined control sequence' error when compiled standalone.
% After all the class is not loaded at this point, therefore no \cls{standalone} macros are yet defined.
% The |\csname|\,\ldots|\endcsname| construct will simple make it equal to |\relax| in this case.
%
% Please note that all code before |\documentclass| is not processed by any of the \opt{subpreamble} options but always simply removed.
% This macro was inspired by the similar macro |\docmute| of the |docmute| package.
%
% \DescribeEnv[<sub-file content>]{standalone}
% The \env{standalone} environment is automatically wrapped around the content of each sub-file when compiled standalone.
% By default it only contains a \env{preview} environment as long the class is not called with the |preview=false| option.
% It can be redefined in the configuration file if required.
% When compiled as part of a main document the \env{standalone} environment does nothing (apart of being a \TeX\ group).
%
% The beamer specific macros and environments are described in section~\ref{sec:beamer}.
%
% \subsection{Configuration}\label{sec:cfg}
% The \cls{standalone} class loads the configuration file \file{standalone.cfg} after the if-switches and options are defined, but before the options are proccessed.
% The default options which should be passed to \pkg{preview} or \cls{beamer} can be set here using:\\[\smallskipamount]
% \hspace*{2\parindent}\Macro\PassOptionsToPackage{<options>}{preview}\\
% \hspace*{2\parindent}\Macro\PassOptionsToClass{<options>}{beamer}\\[\medskipamount]
% The if-switches can be used, but \Macro{ifbeamerstandalone} will always be false because the \opt{beamer}
% option is not yet processed. Such code should be places inside \Macro\AtEndOfClass{\ldots}. 
% Please note that this was handled differently before v1.0, so in old configuration files edited by the user the \Macro\AtEndOfClass must now be added.
%
% \DescribeMacro\standaloneconfig{<key=value, key=value, \ldots>}
% This macro is defined by the class and also provided by the package and allows the configuration of both.
% All package and class options can be used here as well, but are only valid in the current group (i.e.\ global when used in the preamble or config file).
% However, the class options must be prefixed with `|cls/|'. The key=value engine of the \pkg{pgfkeys} package (\pkg{pgf} bundle) is used which is
% described in detail in the |pgfmanual|, part V, section 55 "Key Management".
%
% This macro also allows the definitions of custom class options, e.g.\ to support different image types for conversion.
% For example to define the short-key `|bmp|' to convert to the BMP format use:\\
% \hbox to \linewidth{\hss|\standaloneconfig{cls/bmp/.style={convert={outext={.bmp}}}}|\hss}
%
% ^^A\newpage
% \subsection{Usage Examples}
%
% \def\lstlistingname{Example}
%
% \par\noindent
% \begin{minipage}{\textwidth}\small
% \begin{lstlisting}[language={[latex]tex},gobble=4,frame=lines,caption={Use of \emph{standalone} package.}]
%   % Main file
%   % Real document class:
%   \documentclass{article}
%
%   % Use the 'standalone' package:
%   \usepackage{standalone}
%
%   % Load all packages needed for all sub-files:
%   \usepackage{tikz}
%
%   % Inside the real 'document' environment
%   % read the sub-file with '\input'
%   \begin{document}
%   % ...
%   \begin{figure}
%     \input{subfile}
%     \caption{A subfile}
%   \end{figure}
%   % ...
%   \end{document}
% \end{lstlisting}
% \end{minipage}
%
% \par\noindent
% \begin{minipage}{\textwidth}\small
% \begin{lstlisting}[language={[latex]tex},gobble=4,frame=lines,caption={Use of \emph{standalone} class.}]
%   % A sub-file (e.g. picture) using the 'standalone' class:
%   % Use 'standalone' as document class:
%   \documentclass{standalone}
%
%   % Load packages needed for this TeX file:
%   \usepackage{tikz}
%
%   % Surround TeX code with 'document' environment as usually:
%   \begin{document}
%   % Add your TeX code, e.g. a picture:
%   \begin{tikzpicture}
%     \draw (0,0) rectangle (2,1) node [midway] {Example};
%   \end{tikzpicture}
%   \end{document}
% \end{lstlisting}
% \end{minipage}
%
% \par\noindent
% \begin{minipage}{\textwidth}\small
% \begin{lstlisting}[language={[latex]tex},gobble=4,frame=lines,caption={Effective code if compiled standalone.}]
%   \documentclass{article}
%
%   \newenvironment{standalone}{\begin{preview}}{\end{preview}}
%   % vim: ft=tex
\PassOptionsToPackage{active}{preview}

%   % which by defaults loads:
%   % \PassOptionsToPackage{active,tightpage}{preview}
%   \usepackage{preview}
%
%   \usepackage{tikz}
%
%   \begin{document}
%   \begin{standalone}
%   \begin{tikzpicture}
%     \draw (0,0) rectangle (2,1) node [midway] {Example};
%   \end{tikzpicture}
%   \end{standalone}
%   \end{document}
% \end{lstlisting}
% \end{minipage}
%
% \par\noindent
% \begin{minipage}{\textwidth}\small
% \begin{lstlisting}[language={[latex]tex},gobble=4,frame=lines,caption={Effective code if included in a main document.}]
%   \begingroup
%   \begin{tikzpicture}
%     \draw (0,0) rectangle (2,1) node [midway] {Example};
%   \end{tikzpicture}
%   \endgroup
%   \endinput
% \end{lstlisting}
% \end{minipage}
%
%
%
% \subsection{Support for Beamer Presentations}\label{sec:beamer}
% Presentation can be written in \LaTeX\ using the |beamer| class. Each presentation frame is wrapped in a \env{frame} environment. Overlay effects
% can be added using special macros. This effects result in multiple pages per frame. Pictures with such overlay effects can not be
% compiled standalone using the normal settings. Instead the \cls{standalone} class must load the |beamer| class and wrap the content also in
% a \env{frame} environment while skipping the \env{preview} environment. To activate this settings load the \cls{standalone} class with the |beamer| option.
% Because the \env{frame} environment is quite special (it normally collects all it's content and calls the \cs{frame}) and must also support
% verbatim content it is not easily possible to redefined the |document| environment to include \env{frame}. Also \env{frame} accepts options which |document| doesn't.
% Therefore a second environment called |standaloneframe| is used in the beamer picture files. It will be equal to \env{frame} in standalone mode, but
% without effect otherwise.
%
% \DescribeMacro{\ifstandalonebeamer}
% Both the class and the package provide the if-switch \cs{ifstandalonebeamer}, which can be used to only include code if the file is
% compiled standalone with the |beamer| class option set.
% The switch is set to \cs{iftrue} by the class when loaded with the |beamer| option and always to \cs{iffalse} by the package.
% It can be used to place beamer specific options in the configuration files, which should be skipped for non-beamer standalone files.
% Starting from v1.0 of \cls{standalone} this switch must be placed inside \Macro\AtEndOfClass{\ldots} when used inside the config file.
%
% \begin{DescribeEnv}{standaloneframe}
%   !\begingroup\oargstyle!<< <overlay specification> >>!\endgroup!
%   ['<'<default overlay spec>'>']!\\\hspace*{7em}!
%   [<options>]{<optional frame title>}{<optional frame subtitle>}
%   \relax\MacroArgs<code with beamer overlays>
% \end{DescribeEnv}
% The \env{standaloneframe} environment must be used in sub-file holding beamer overlay code.
% It is only defined when the class is called with the |beamer| option and acts as a replacement
% of the \env{frame} environment of beamer when compiled standalone.
% All optional arguments of \env{frame} are supported but most might not be useful for normal sub-files.
% When compiled as part of a main document it does nothing except of gobbling its arguments.
%
%
%
% \par\bigskip\noindent
% \begin{minipage}{\textwidth}\noindent\small
% \begin{lstlisting}[language={[latex]tex},gobble=4,frame=lines,caption={Use of \texttt{standalone} class with \texttt{beamer} option.}]
%   % Use of 'standalone' class with a beamer overlay:
%   \documentclass[beamer]{standalone}
%   % Load packages needed for this TeX file:
%   \usepackage{tikz}
%
%   % Surround TeX code with 'document' environment as usually:
%   \begin{document}
%   \begin{standaloneframe}[options] % e.g. 'fragile'
%    % Add your TeX code:
%     \only<1>{ One }%
%     \only<2>{ Two }%
%   \end{standaloneframe}
%   \end{document}
% \end{lstlisting}
% \end{minipage}
%
% \par\bigskip\noindent
% \begin{minipage}{\textwidth}\noindent\small
% \begin{lstlisting}[language={[latex]tex},gobble=4,frame=lines,caption={Effective beamer code if compiled standalone.}]
%   \documentclass{beamer}
%   % vim: ft=tex
\PassOptionsToPackage{active}{preview}

%
%   \usepackage{tikz}
%
%   \begin{document}
%   \begin{frame}[your options]
%     \only<1>{ One }%
%     \only<2>{ Two }%
%   \end{frame}
%   \end{document}
% \end{lstlisting}
% \end{minipage}
%
% \par\bigskip\noindent
% \begin{minipage}{\textwidth}\noindent\small
% \begin{lstlisting}[language={[latex]tex},gobble=4,frame=lines,caption={Effective code if included in a beamer presentation.}]
%   \begingroup
%     \only<1>{ One }%
%     \only<2>{ Two }%
%   \endgroup
%   \endinput
% \end{lstlisting}
% \end{minipage}
%
% \subsection{Simple TeX File}
% \noindent
% The simple |standalone.tex| file will set the \cs{ifstandalone} switch to true when compiled standalone but to false
% when loaded after any |\documentclass| macro, as long the switch isn't defined yet.
% It must be used if the switch is required is intended to be used before the |\documentclass| of a standalone file.
%
% \noindent
% \begin{minipage}{\textwidth}\noindent\small
% \begin{lstlisting}[language={[latex]tex},gobble=4,frame=lines,caption={Usage of 'standalone.tex'.}]
%   % \iffalse meta-comment
%
% Copyright (c) 2010 by Martin Scharrer <martin@scharrer-online.de>
% -----------------------------------------------------------------
%
% This work may be distributed and/or modified under the
% conditions of the LaTeX Project Public License, either version 1.3c
% of this license or (at your option) any later version.
% The latest version of this license is in
%
%   http://www.latex-project.org/lppl.txt
%
% and version 1.3c or later is part of all distributions of LaTeX
% version 2008/05/04 or later.
%
% This work has the LPPL maintenance status `maintained'.
%
% The Current Maintainer of this work is Martin Scharrer.
%
% This work consists of the files standalone.dtx, standalone.ins
% and the derived file standalone.sty.
%
% $Id$
% \fi
%
% \iffalse
%<cls|sty|cfg|tex>% $Id$
%<cls|sty>\NeedsTeXFormat{LaTeX2e}
%<cls>\ProvidesClass{standalone}   [2010/03/20 v0.1 Class to compile TeX sub-files standalone]
%<sty>\ProvidesPackage{standalone} [2010/03/20 v0.1 Package to include TeX sub-files which use the 'standalone' class]
%<tex>\ProvidesFile{standalone.tex}[2010/03/20 v0.1 Provides if-switch to show if file is compiled standalone]%
%<cfg>\ProvidesFile{standalone.cfg}[2010/03/20 v0.1 Default configuration file for 'standalone' class and package]%
%<*driver>
\NeedsTeXFormat{LaTeX2e}
\ProvidesFile{standalone.dtx}[2010/03/20 v0.1 DTX File for 'standalone' cls/sty/tex]
\documentclass{ltxdoc}
\usepackage{ifpdf}
\usepackage{hyperref}
\usepackage{tabularx}
\usepackage{array}
\usepackage{flafter,fnpos}
\usepackage{booktabs}
\usepackage{amsmath}
\usepackage{placeins}
\makeFNbottom
\makeFNbelow

\EnableCrossrefs
%\CodelineIndex
\RecordChanges
%\OnlyDescription
\begin{document}
  \DocInput{\jobname.dtx}
  \PrintChanges
  %\newpage\PrintIndex
\end{document}
%</driver>
% \fi
%
% \CheckSum{0}
%
% \CharacterTable
%  {Upper-case    \A\B\C\D\E\F\G\H\I\J\K\L\M\N\O\P\Q\R\S\T\U\V\W\X\Y\Z
%   Lower-case    \a\b\c\d\e\f\g\h\i\j\k\l\m\n\o\p\q\r\s\t\u\v\w\x\y\z
%   Digits        \0\1\2\3\4\5\6\7\8\9
%   Exclamation   \!     Double quote  \"     Hash (number) \#
%   Dollar        \$     Percent       \%     Ampersand     \&
%   Acute accent  \'     Left paren    \(     Right paren   \)
%   Asterisk      \*     Plus          \+     Comma         \,
%   Minus         \-     Point         \.     Solidus       \/
%   Colon         \:     Semicolon     \;     Less than     \<
%   Equals        \=     Greater than  \>     Question mark \?
%   Commercial at \@     Left bracket  \[     Backslash     \\
%   Right bracket \]     Circumflex    \^     Underscore    \_
%   Grave accent  \`     Left brace    \{     Vertical bar  \|
%   Right brace   \}     Tilde         \~}
%
%
% \changes{v0.1}{2010/03/20}{First released version}
%
% \GetFileInfo{\jobname.dtx}
%
% \DoNotIndex{\newcommand,\newenvironment,\def,\edef,\xdef,\DeclareRobustCommand}
% \DoNotIndex{\expandafter,\if,\else,\fi,\ifnum,\ifx,\let,\global,\long}
% \DoNotIndex{\newcounter,\newcount,\message,\meaning,\noexpand,\relax,\value}
% \DoNotIndex{\setcounter,\addtocounter,\advance,\afterassignment,\AtEndOfPackage}
% \DoNotIndex{\ProvidesPackage,\providecommand,\RequirePackage,\empty,\begin,\end}
% \DoNotIndex{\begingroup,\bgroup,\egroup,\endgroup,\csname,\endcsname,\@tempa,\@tempb}
% \DoNotIndex{\ignorespaces,\lccode,\sffamily,\@gobble,\@ifundefined,\@for,\or}
% \DoNotIndex{\@firstoftwo,\@ifnextchar,\@namedef,\@nameuse,\@secondoftwo}
% \DoNotIndex{\@temptokena,\toks@,\BODY,\do,\g@addto@macro,\lowercase,\uppercase,\the}
%
% \ifpdf
% \hypersetup{%
%   pdfauthor   = {Martin Scharrer <martin@scharrer-online.de>},
%   pdftitle    = {The standalone class and package},
%   pdfsubject  = {Documentation of LaTeX class and package 'standalone'},
%   pdfkeywords = {standalone, LaTeX, TeX}
% }%
% \fi
% \clearpage
% \null
% \vspace*{-2em}
% \begin{center}
%   {\LARGE\sffamily The \emph{standalone} Class and Package\\}
%   {\large Martin Scharrer \\\normalsize 
%   \url{martin@scharrer-online.de}\\[.8ex]
%   \url{http://www.ctan.org/pkg/standalone/}\\[1.2ex]}
%   {\large Version \fileversion\ -- \filedate}\\
% \end{center}
% \vspace{1.2em}%
%
% \section{Introduction}
%
% \StopEventually{}
% \FloatBarrier
% \clearpage
%
% \section{Implementation}
%
% \subsection{The Package File}
% \iffalse
%<*sty>
% \fi
%    \begin{macrocode}
\expandafter\newif\csname ifstandalone\endcsname
\standalonefalse
%    \end{macrocode}
%
%  \begin{macro}{\sa@documentclass}
%    \begin{macrocode}
\newcommand{\sa@documentclass}[2][]{%
  \let\document\sa@document
  \let\enddocument\sa@enddocument
  \expandafter\sa@@documentclass
}
%    \end{macrocode}
%  \end{macro}
%
%  \begin{macro}{\sa@@documentclass}
%    \begin{macrocode}
\def\sa@@documentclass{%
  \begingroup\def\sa@gobbleto{document}\sa@gobble
}
%    \end{macrocode}
%  \end{macro}
%
%  \begin{macro}{\sa@gobble}
%    \begin{macrocode}
\long\def\sa@gobble#1\begin#2{%
  \def\@tempa{#2}%
  \ifx\@tempa\sa@gobbleto
    \def\next{\expandafter\endgroup\expandafter\begin\expandafter{\sa@gobbleto}}%
  \else
    \def\next{\sa@gobble}%
  \fi
  \next
}
%    \end{macrocode}
%  \end{macro}
%
%    \begin{macrocode}
\expandafter\ifx\csname standalone\endcsname\relax
  \newenvironment{standalone}[1][]{}{}
\fi
%    \end{macrocode}
%
%  \begin{macro}{\sa@orig@document}
%    \begin{macrocode}
\let\sa@orig@document\document
%  \end{macro}
%
%  \begin{macro}{\sa@orig@enddocument}
%    \begin{macrocode}
\let\sa@orig@enddocument\enddocument
%    \end{macrocode}
%  \end{macro}
%
%  \begin{macro}{\document}
%    \begin{macrocode}
\def\document{%
  \sa@orig@document
  \let\documentclass\sa@documentclass
  \ignorespaces
}
%    \end{macrocode}
%  \end{macro}
%
%  \begin{macro}{\sa@document}
%    \begin{macrocode}
\def\sa@document{%
  \sa@atbegindocument
}
%    \end{macrocode}
%  \end{macro}
%
%  \begin{macro}{\sa@enddocument}
%    \begin{macrocode}
\def\sa@enddocument{%
  \sa@atenddocument
  \aftergroup\sa@@enddocument
}
%    \end{macrocode}
%  \end{macro}
%
%  \begin{macro}{\sa@atbegindocument}
%    \begin{macrocode}
\def\sa@atbegindocument{%
  \ignorespaces
}%
%    \end{macrocode}
%  \end{macro}
%
%  \begin{macro}{\sa@atenddocument}
%    \begin{macrocode}
\def\sa@atenddocument{%
  \unskip
}%
%    \end{macrocode}
%  \end{macro}
%
%  \begin{macro}{\sa@@enddocument}
%    \begin{macrocode}
\def\sa@@enddocument{%
  %\let\document\sa@orig@document
  \let\enddocument\sa@orig@enddocument
  \endinput
}
%    \end{macrocode}
%  \end{macro}
%
%  \begin{macro}{\sa@processpreamble}
%    \begin{macrocode}
\def\sa@processpreamble{%
  \renewcommand\usepackage[2][]{%
    \message{^^J%
      INFO: Sub-file requires the following package(s):^^J
      \space\space[##1]{##2}^^J%
    }%
  }%
  \let\RequirePackage\usepackage
}
%    \end{macrocode}
%  \end{macro}
%
%    \begin{macrocode}
%%\def\sa@@documentclass{\sa@processpreamble}
%    \end{macrocode}
%
% \iffalse
%</sty>
% \fi
%
% \subsection{The Class File}
% \iffalse
%<*cls>
% \fi
%    \begin{macrocode}
\def\sa@classoptionslist{}
\RequirePackage{kvoptions}
\SetupKeyvalOptions{prefix=sa@}
\DeclareBoolOption[true]{preview}
%    \end{macrocode}
%
%  \begin{macro}{\sa@cls@document}
%    \begin{macrocode}
\def\sa@cls@document{\ifsa@preview\preview\fi}
%    \end{macrocode}
%  \end{macro}
%
%  \begin{macro}{\sa@cls@enddocument}
%    \begin{macrocode}
\def\sa@cls@enddocument{\ifsa@preview\endpreview\fi}
%    \end{macrocode}
%  \end{macro}
%
%    \begin{macrocode}
\DeclareVoidOption{beamer}{%
  \def\sa@class{beamer}%
  \sa@previewfalse
  \newenvironment{standalone}[1][]{%
  \begin{frame}[fragile,environment=standalone]}{\end{frame}}%
}
\DeclareStringOption[article]{class}
\DeclareStringOption[]{classoptions}
\DeclareStringOption[]{frameoptions}
\DeclareDefaultOption{%
  \xdef\sa@classoptionslist{\sa@classoptionslist,\CurrentOption}%
}
% vim: ft=tex
\PassOptionsToPackage{active}{preview}

\ProcessKeyvalOptions*\relax
%    \end{macrocode}
%
%    \begin{macrocode}
\let\@classoptionslist\sa@classoptionslist
\xdef\@tempa{[\sa@classoptions]{\sa@class}}
\expandafter\LoadClass\@tempa
%    \end{macrocode}
%
%    \begin{macrocode}
\ifsa@preview
  \RequirePackage{preview}
\fi
%    \end{macrocode}
%
%    \begin{macrocode}
\RequirePackage{standalone}[2010/03/20]
\standalonetrue
%    \end{macrocode}
%
%  \begin{macro}{\document}
%    \begin{macrocode}
\def\document{%
  \sa@orig@document
  \let\documentclass\sa@documentclass
  \sa@cls@document
}
%    \end{macrocode}
%  \end{macro}
%
%  \begin{macro}{\enddocument}
%    \begin{macrocode}
\def\enddocument{%
  \sa@cls@enddocument
  \sa@orig@enddocument
}
%    \end{macrocode}
%  \end{macro}
%
% \iffalse
%</cls>
% \fi
%
% \subsection{Simple TeX File}
% \iffalse
%<*tex>
% \fi
%    \begin{macrocode}
\expandafter\ifx\csname ifstandalone\endcsname\relax
%    \end{macrocode}
%    \begin{macrocode}
\expandafter\newif\csname ifstandalone\endcsname
%    \end{macrocode}
%    \begin{macrocode}
\expandafter\ifx\csname @twoclasseserror\endcsname\documentclass
\else
  \standalonetrue
\fi
%    \end{macrocode}
%    \begin{macrocode}
\fi
%    \end{macrocode}
% \iffalse
%</tex>
% \fi
%
% \subsection{Config File}
% \iffalse
%<*cfg>
% \fi
%    \begin{macrocode}
\PassOptionsToPackage{active,tightpage}{preview}%
%    \end{macrocode}
% \iffalse
%</cfg>
% \fi
%
% \iffalse
%<cfg>% vim: ft=tex
% \fi
% \Finale
% \endinput
 % use before any '\documentclass'
%   \ifstandalone
%     % Used only if compiled standalone
%   \fi
% \end{lstlisting}
% \end{minipage}
%
% \subsection{Usage with \texttt{svn-multi} keywords}\label{sec:svnmulti}
% If the version control package \texttt{svn-multi} is used, the keyword macros (|\svnid| or |\svnidlong|)
% need to be placed after the |\begin{document}| to be taken into account by the main document.
% The \texttt{svn-multi} package must be loaded by the sub-file preamble to avoid compile errors in standalone mode.
% Alternative, if the keywords are not required in this mode, they can be surrounded by |\ifstandalone\else|\dots|\fi|.
%
% \subsection{Convert to Image}
% \textbf{This feature is new and not fully tested on every platform yet. Please do not hesitate to contact the author by e-mail about issues or suggestions.}
%
% \par\medskip\noindent
% The compiled output PDF (DVI/PS support pending) can be converted to an image (recommended is PNG, but JPG or other are also supported) with
% the help of an external application. The resulting image is useful for web-publishing or importing the standalone TeX picture into a
% software with can't handle PDF/(E)PS.
%
% \subsubsection{Conversion Options}
% The \opt*{convert} option controls the conversion feature of \cls{standalone}. It takes an optional value which can be used to specify
% sub-options. Without a value the conversion defaults (PNG, density 300dpi) are used. The conversion is disabled by default.
% Using the \opt{convert} option with any value other than |false| will enabled it.
% Table~\ref{tab:convopt} lists the normal conversion options for the conversation settings.
% Table~\ref{tab:advconvopt} lists the more advanced options which can be used to modify the conversation command or to add support for other
% output formats. The \opt{border} option might be used in addition to adjust the border which will affect the resulting image.
%
% For the common output formats PNG (recommended for rastered vector graphics), JPG and GIF the class options \opt*{png}, \opt*{jpg} and \opt*{gif}
% are also defined. They simply call \opt{convert} with the correct settings of \opt{outext} and pass an optional value to it. They can also be used
% inside the value of \opt{convert}. Further defined shortkeys are \opt{png72}, \opt{png150} and \opt{png300} which also set the density to the given number.
%
% \par\medskip
% \noindent\textbf{Examples:}\\[\smallskipamount]
% PDF/PS is rastered with 600x100dpi and then converted to JPG:\\*
% |  \documentclass[convert={density=600x100,jpg}]{standalone}|\\[\smallskipamount]
% Produces BMP with 400x400px\\*
% |  \documentclass[convert={outext=.bmp,size=400}]{standalone}|\\[\smallskipamount]
% Short-key to build a PNG:\\*
% |  \documentclass[png]{standalone}|\\[\smallskipamount]
% Short-key to build a PNG, with extra settings:\\*
% |  \documentclass[png={size=200x400,density=600}]{standalone}|\\[\smallskipamount]
%
%
% \subsubsection{DVI Support}
% The conversion option is mainly developed with PDF output in mind. To the knowledge of the author
% DVI output requires the conversion to PostScript (PS) before it can be converted to an image.
% This is done by and pre-command which is executed before the actual conversion command. It defaults to `|dvips \jobname|' or is empty if PDF output is detected.
%
% \subsubsection{Conversion Programs}\label{sec:convprogs}
% The actual conversion is done using an external program. This program is called by the \cls{standalone} class direct after the class options are processed.
% Then the output file (PDF or DVI) created by the last run is not yet opened and therefore still in its old condition, so that it can be converted.
% By default the execution of external program is restricted for \LaTeX\ to a very limited set of programs. This is for security reasons to prevent
% documents a user might have received from another person to cause harm. In order to run the image conversation program the `{\bfseries|-shell-escape|}'
% command line argument of the \LaTeX{} compiler must be used (for TeXLive; MikTeX uses officially `|--enable-write18|' but also understand the other one).
% Another possibility is to add the used conversation executable to list of allowed commands.
% For TeXLive this is done by adding it to the |shell_escape_commands| variable in the |texmf.cnf| file.
%
% By default the conversion program of Image Magick (\url{http://www.imagemagick.org/}) is used, which is freely available for Unix/Linux, Mac and MS Windows.
% Under Ubuntu Linux it can be installed using `|sudo apt-get install imagemagick|'.
% The conversion executable is simply called `|convert|'. However, there is another program with the same name provided by MS Windows itself which
% converts old FAT filesystems to NTFS! It has been suggested to rename the Image Magick executable to `|imgconvert|' instead.
% By default \cls{standalone} uses `|imgconvert|' as executable and only uses `|convert|' if it is found at `|/usr/bin/convert|' or `|/usr/local/bin/convert|'
% which indicates a Unix/Linux OS where the `|convert|' name is save to use.
% The executable name can be change manually using the `|convertexe|' conversion option or by using\\
% \centerline{\Macro\standaloneconfig{'cls/convert/convertexe='{convert}}}
% in the configuration file `|standalone.cfg|'.
%
% Another conversion program is Ghostscript (\url{http://pages.cs.wisc.edu/~ghost/}) which is a very common PostScript interpreter which also supports PDF.
% Under Ubuntu Linux it is most likely already installed but otherwise can be installed using `|sudo apt-get install ghostscript|' or
% `|sudo apt-get install gs|'.
% It can convert both to various output formats and is freely available for Unix/Linux, Max OS X and MS Windows.
% It requires to set the correct output device which is not always fully identical to the output format (e.g.\ `|png16m|' for a PNG (with 16 million colors)).
% The devices for PNG and JPG are already configured.
% Other devices can be directly set using the \opt{gsdevice} conversion option or by defining it using:
% ^^A\centerline
% {\footnotesize\Macro\standaloneconfig{'cls/convert/gsdevice/ext'<.extension>'/.initial='{<device>}}}
% in the configuration file `|standalone.cfg|'.
%
% \begin{table}
% \belowcaptionskip=\abovecaptionskip
% \caption[Conversion Options]{Conversion Options (to be used in the value of \opt{convert} class option)}\label{tab:convopt}
% \hbox to \textwidth{\hss%
% \begin{tabular}{>{\ttfamily}lp{\dimexpr.7\textwidth\relax}l}
%  \toprule
%  \multicolumn{1}{l}{Sub-Option}  &  Description & Default value \\
%  \midrule
%   (no value) &  Conversion enabled with default settings.  & ./. \\
%     true     &  Conversion enabled (with default settings if no other options are given). & (no value) \\
%     false    &  Conversion disabled.  & (no value) \\
%     newer    &  Convert only if the |.tex| file is newer than the output file. & (no value/default) \\
%     force    &  Force conversion (opposite of "newer")  & (no value) \\
%  \midrule
%     density  &  Sets the density in dots-per-inch (dpi). Can be a single numerical value or `\meta{X}|x|\meta{Y}'.  & |300| \\
%     size     &  Sets the size of the image. Can be a single numerical value or `\meta{X}|x|\meta{Y}'. If empty the size is determined by the density setting and the size of the PDF. & (empty) \\
%     inext    &  Input file extension including the leading dot. & |\ifpdf.pdf\else.ps\fi| \\
%     inname   &  Name base of input file (i.e.\ file name without extension). & |\jobname| \\
%     infile   &  Input file name. & |\inname\inext| \\
%     outext   &  Output file extension including the leading dot. & |.pdf| \\
%     outname  &  Name base of output file (i.e.\ file name without extension). & |\inname| \\
%     outfile  &  Output file name. & |\outname\outext| \\
%  \bottomrule\noalign{\smallskip}
%  \multicolumn{3}{l}{%
%   \strut Note: the settings (except 'true' and 'false') can also be used as macros in other settings.
%   }%
% \end{tabular}
% \hss}
% \end{table}
% \begin{table}
% \belowcaptionskip=\abovecaptionskip
% \caption{Advanced Conversion Options}\label{tab:advconvopt}
% \hbox to \textwidth{\hss%
% \begin{tabular}{>{\ttfamily}lp{\dimexpr.6\textwidth\relax}l}
%  \toprule
%  \multicolumn{1}{l}{Sub-Option}  &  Description & Default value \\
%  \midrule
%     command  &  Command line used for conversion. & (see |imagemagick|) \\
%     imagemagick & Sets the convert command to use Image Magick: \newline\bigskip
%                   \hspace*{1em}\hbox{\vtop{\hbox{\strut|command={\convertexe\space -density \density\space |\vrule height 2\ht\strutbox width 0pt{}\egroup
%                                      \hbox\bgroup\strut|         \infile\space \ifx\size\empty\else |\egroup
%                                      \hbox\bgroup\strut|         -resize \size\fi\space -quality 90 \outfile}|}}\hss} \\
%     convertexe  & Name of the executable of Image Magick. & (see section~\ref{sec:convprogs})  \\
%     ghostscript & Sets the convert command to use Ghostscript: \newline\bigskip
%                   \hspace*{1em}\hbox{\vtop{\hbox{\strut|command={gs -dSAFER -dBATCH -dNOPAUSE -sDEVICE=\gsdevice\space|\vrule height 2\ht\strutbox width 0pt{}\egroup
%                                      \hbox\bgroup\strut|         -r\density\space -sOutputFile=\outfile\space \infile}|}}\hss} \\
%     gs          & Alias for |ghostscript| \\
%     precommand  & Command to be executed before the actual conversion command.  & \hbox{\vtop{\hbox{\strut|\ifpdf\else|}\hbox{\strut{|  dvips \jobname \fi|}}}} \\
%     gsdevice    & The output device to be used for ghostscript. Already set up for PNG and JPG output. & \hbox{\vtop{\hbox{value of `|gsdevice/ext\outext|'\strut}{\hbox{\strut if defined, otherwise |\outext|}\hbox{\strut without the dot}}}} \\
%       onfailure & Sets if an type of `message' which should be triggered on conversion failure:  |error| (default), |warning|, |info| or |ignore|. & |error| \\
%  \bottomrule
% \end{tabular}
% \hss}
% \end{table}
%
% \StopEventually{}
% \clearpage
% \section{Implementation}
%
% \iffalse
%<@standalone.cls>
% \fi
%
% \iffalse
%<@standalone.sty>
% \fi
%
% \iffalse
%<@standalone.tex>
% \fi
%
% \iffalse
%<@standalone.cfg>
% \fi
%
% \Finale
\endinput
