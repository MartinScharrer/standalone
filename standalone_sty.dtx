%
% \iffalse
%<*sty>
% \fi
% \subsection{The Package File}
%
%    \begin{macrocode}
\NeedsTeXFormat{LaTeX2e}
\RequiresPackage{svn-prov}[2010/08/24]
\ProvidesPackageSVN{$Id$}[v0.4][Package to include TeX sub-files with preambles]
%    \end{macrocode}
%
% The package file is to be loaded by a main document which includes |standalone| sub-files.
% It is also loaded by the |standalone| class to share code. The class then redefines certain macros.
%
% \subsubsection{If-Switches}
%
% \begin{macro}{\ifstandalone}
% Declare |standalone| if-switch and set it to false. The class will set it to true.
% The |\csname| trickery is used to avoid issues if the switch was already defined.
%    \begin{macrocode}
\expandafter\newif\csname ifstandalone\endcsname
\standalonefalse
%    \end{macrocode}
% \end{macro}
%
% \begin{macro}{\ifstandalonebeamer}
% \changes{v0.2}{2010/03/23}{New macro}
% Make sure that |standalonebeamer| if-switch is defined and set it to false.
% If the class was loaded beforehand with the |beamer| option it is already defined as true.
% The |\csname| trickery is used to avoid issues if the switch was already defined.
%    \begin{macrocode}
\@ifundefined{ifstandalonebeamer}{%
\expandafter\newif\csname ifstandalonebeamer\endcsname
\standalonebeamerfalse
}{}%
%    \end{macrocode}
% \end{macro}
%
% \begin{macro}{\onlyifstandalone}
% \changes{v0.3}{2010/03/26}{New macro}
% Macro version of |\ifstandalone|. The |{ }| around the argument protects the content from the package etc. scanners.
% Only defined if not already defined by the class, in the case of a |standalone| file included other |standalone| files.
%    \begin{macrocode}
\@ifundefined{onlyifstandalone}
  {\let\onlyifstandalone\@gobble}
  {}
%    \end{macrocode}
% \end{macro}
%
%
% \begin{macro}{\ifsa@subpreambles}
% \begin{macro}{\ifsa@sortsubpreambles}
% \begin{macro}{\ifsa@printsubpreambles}
% The if-switches for the options.
%    \begin{macrocode}
\newif\ifsa@subpreambles
\newif\ifsa@sortsubpreambles
\newif\ifsa@printsubpreambles
%    \end{macrocode}
% \end{macro}
% \end{macro}
% \end{macro}
%
% \subsubsection{Options}
%    \begin{macrocode}
\RequirePackage{ifpdf}
\RequirePackage{pgfopts}
%
\pgfkeys{/standalone/.cd,
  subpreambles/.is if=sa@subpreambles,
  subpreambles/.default=true,
%
  sort/.is choice,
  sort/true/.code={%
    \sa@sortsubpreamblestrue
    \sa@subpreamblestrue
  },
  sort/false/.code={%
    \sa@sortsubpreamblesfalse
  },
  sort/.default=true,
%
  print/.is choice,
  print/true/.code={%
    \sa@printsubpreamblestrue
    \sa@subpreamblestrue
  },
  print/false/.code={%
    \sa@printsubpreamblesfalse
  },
  print/.default=true,
%
  comments/.is choice,
  comments/true/.code=%
  },
  comments/false/.code={%
    \def\sa@percent{}%
  },
  comments/.default=true,
  nocomments/.style={/standalone/comments=false},
  nocomments/.value forbidden,
%
  mode/.is choice,
  mode/none/.code={%
    \let\sa@mode\relax
  },
  mode/pdf|tex/.code={%
    \def\sa@mode{0}%
  },
  mode/tex/.code={%
    \def\sa@mode{1}%
  },
  mode/pdf/.code={%
    \def\sa@mode{2}%
  },
  mode/build/.code={%
    \def\sa@mode{3}%
  },
  mode/buildnew/.code={%
    \def\sa@mode{4}%
  },
  mode=tex,
%
  external/ext/.store in=\sa@graphicext,
  external/ext={\ifpdf .pdf\else .eps\fi},
%
  external/cmd/.store in=\sa@compilecmd,
  external/cmd={\ifpdf pdf\fi latex --interaction=batchmode \image},
}
\ProcessPgfOptions{/standalone}
%    \end{macrocode}
%
% In non-sorted print mode comments are preserved by default.
%    \begin{macrocode}
\ifsa@printsubpreambles
  \ifsa@sortsubpreambles\else
    \@ifundefined{sa@percent}%
    }{}%
  \fi
\fi
%    \end{macrocode}
%
% \begin{macro}{\standaloneconfig}
%    \begin{macrocode}
\@ifclassloaded{standalone}{}{%
\newcommand*{\standaloneconfig}{\pgfqkeys{/standalone}}
}
%    \end{macrocode}
% \end{macro}
%
%
% The \pkg{currfile} package is used to get the file paths of the included sub-files.
%    \begin{macrocode}
\RequirePackage{currfile}
%    \end{macrocode}
%
% \subsubsection{Processing of Sub-Preambles}
%
%    \begin{macrocode}
\ifsa@subpreambles
%    \end{macrocode}
%
% \begin{macro}{\sa@out}
% Write handle.
%    \begin{macrocode}
\newwrite\sa@out
%    \end{macrocode}
% \end{macro}
%
% \begin{macro}{\sa@write}
% Helper macro for file output.
%    \begin{macrocode}
\def\sa@write{\immediate\write\sa@out}%
%    \end{macrocode}
% \end{macro}
%
%
%    \begin{macrocode}
\ifsa@printsubpreambles
%    \end{macrocode}
%
% \begin{macro}{\sa@removeonlyifstandalone}
% Scans for |\onlyifstandalone| and removes it argument.
%    \begin{macrocode}
\long\def\sa@removeonlyifstandalone#1\onlyifstandalone{%
  \g@addto@macro\sa@preamble{#1}%
  \@ifnextchar\sa@endmarker
    {\@gobble}%
    {\expandafter\sa@gobbleeol\expandafter\sa@removeonlyifstandalone\expandafter^^J\@gobble}%
}
%    \end{macrocode}
% \end{macro}
%
%    \begin{macrocode}
\fi
%    \end{macrocode}
%
%
% \subsubsection{Sorting of package options}
%
% Macros only needed for this mode are defined inside the |\if...|
% to save memory otherwise.
%
%    \begin{macrocode}
\ifsa@sortsubpreambles
%    \end{macrocode}
%
% \begin{macro}{\sa@usepackagewithoutoptions}
% Simply calls the original |\usepackage| while skipping the optional
% argument with potential package options.
%    \begin{macrocode}
\newcommand{\sa@usepackagewithoutoptions}[2][]{%
  \sa@orig@usepackage{#2}%
}
%    \end{macrocode}
% \end{macro}
%
% \begin{macro}{\sa@endmarker}
% Unique end marker. Will not be expanded.
%    \begin{macrocode}
\def\sa@endmarker{%
  \@gobble{sa@endmarker}%
}
%    \end{macrocode}
% \end{macro}
%
%    \begin{macrocode}
\ifsa@printsubpreambles
%    \end{macrocode}
%
%
% In sorted print mode all collected package etc. information is printed into the output file, followed
% by the reduced sub-preambles.
%    \begin{macrocode}
\AtEndDocument{%
  \sa@write{\@percentchar\space Packages required by sub-files:}%
  \expandafter\@for\expandafter\pkg\expandafter:\expandafter=\sa@collpkgs\do{%
    \ifx\pkg\empty\else
      \sa@write{%
        \string\usepackage%
        \expandafter\ifx\csname sa@pkgopts@\pkg\endcsname\empty\else%
          [\csname sa@pkgopts@\pkg\endcsname]%
        \fi
        {\pkg}%
        \expandafter\ifx\csname sa@pkgdate@\pkg\endcsname\relax\else%
          [\csname sa@pkgdate@\pkg\endcsname]%
        \fi
        }%
    \fi
  }%
  \ifx\sa@collpgflibs\empty\else
  \sa@write{^^J\@percentchar\space PGF libraries required by sub-files:}%
  \expandafter\@for\expandafter\lib\expandafter:\expandafter=\sa@collpgflibs\do{%
    \ifx\lib\empty\else
      \sa@write{\string\usepgflibrary{\lib}}%
    \fi
  }%
  \fi
  \ifx\sa@colltikzlibs\empty\else
  \sa@write{^^J\@percentchar\space TikZ libraries required by sub-files:}%
  \expandafter\@for\expandafter\lib\expandafter:\expandafter=\sa@colltikzlibs\do{%
    \ifx\lib\empty\else
      \sa@write{\string\usetikzlibrary{\lib}}%
    \fi
  }%
  \fi
  \ifx\sa@colltikztiminglibs\empty\else
  \sa@write{^^J\@percentchar\space TikZ-Timing libraries required by sub-files:}%
  \expandafter\@for\expandafter\lib\expandafter:\expandafter=\sa@colltikztiminglibs\do{%
    \ifx\lib\empty\else
      \sa@write{%
        \string\usetikztiminglibrary%
        \expandafter\ifx\csname sa@tikztimingopts@\lib\endcsname\empty\else%
          [\csname sa@tikztimingopts@\lib\endcsname]%
        \fi
        {\lib}%
        \expandafter\ifx\csname sa@tikztimingdate@\lib\endcsname\relax\else%
          [\csname sa@tikztimingdate@\lib\endcsname]%
        \fi
        }%
    \fi
  }%
  \fi
  \sa@write{\expandafter\unexpanded\expandafter{\sa@preamble}}%
  \message{^^JPackage 'standalone' INFO: See file '\jobname.stp' for list of sub-preambles.^^J}%
  \immediate\closeout\sa@out
}
%    \end{macrocode}
%
% \begin{macro}{\sa@removepackages}
% Scans for |\usepackage|.
%    \begin{macrocode}
\long\def\sa@removepackages#1\usepackage{%
  \sa@removepgflibs#1\usepgflibrary\sa@endmarker
  \@ifnextchar\sa@endmarker
    {\@gobble}%
    {\sa@sortpackages}%
}
%    \end{macrocode}
% \end{macro}
%
% \begin{macro}{\sa@removepgflibs}
% Scans for |\usepgflibrary|.
%    \begin{macrocode}
\long\def\sa@removepgflibs#1\usepgflibrary{%
  \sa@removetikzlibs#1\usetikzlibrary\sa@endmarker
  \@ifnextchar\sa@endmarker
    {\@gobble}%
    {\sa@sortpgflibs}%
}
%    \end{macrocode}
% \end{macro}
%
% \begin{macro}{\sa@removetikzlibs}
% Scans for |\usetikzlibrary|.
%    \begin{macrocode}
\long\def\sa@removetikzlibs#1\usetikzlibrary{%
  \sa@removetikztiminglibs#1\usetikztiminglibrary\sa@endmarker
  \@ifnextchar\sa@endmarker
    {\@gobble}%
    {\sa@sorttikzlibs}%
}
%    \end{macrocode}
% \end{macro}
%
% \begin{macro}{\sa@removetikztiminglibs}
% Scans for |\usetikztiminglibrary|.
%    \begin{macrocode}
\long\def\sa@removetikztiminglibs#1\usetikztiminglibrary{%
  \sa@removeonlyifstandalone#1\onlyifstandalone\sa@endmarker
  \@ifnextchar\sa@endmarker
    {\@gobble}%
    {\sa@sorttikztiminglibs}%
}
%    \end{macrocode}
% \end{macro}
%
% \begin{macro}{\sa@sortpackage}
% Reads \cs{usepackage} arguments and stores them away. A list of all packages is compiled. Every package is only
% added once and has also a list of options used, also only saved once.
% If package dates are requested then the highest one is stored. Trailing newlines are removed.
%    \begin{macrocode}
\def\sa@collpkgs{}%
\newcommand\sa@sortpackages[2][]{%
  \@ifnextchar[%]
    {\sa@@sortpackages{#1}{#2}}%
    {\sa@@sortpackages{#1}{#2}[]}%
}
\def\sa@@sortpackages#1#2[#3]{%
  \@for\pkg:=#2\do {%
    \@ifundefined{sa@pkgopts@\pkg}%
      {%
        \expandafter\g@addto@macro\expandafter\sa@collpkgs\expandafter{\expandafter,\pkg}%
        \global\@namedef{sa@pkgopts@\pkg}{#1}%
        \global\@namedef{sa@pkgopt@\pkg @}{}%
        \ifx\relax#1\relax\else
          \@for\opt:=#1\do{\global\@namedef{sa@pkgopt@\pkg @\opt}{}}%
        \fi
      }%
      {%
        \ifx\relax#1\relax\else
          \@for\opt:=#1\do{%
            \@ifundefined{sa@pkgopt@\pkg @\opt}%
              {%
                \expandafter\g@addto@macro\csname sa@pkgopts@\pkg\expandafter\endcsname\expandafter{\expandafter,\opt}%
                \global\@namedef{sa@pkgopt@\pkg @\opt}{}%
              }{}%
          }%
        \fi
      }%
    \ifx\relax#3\relax\else
    \@ifundefined{sa@pkgdate@\pkg}%
      {\global\@namedef{sa@pkgdate@\pkg}{#3}}%
      {%
        \ifnum\expandafter\expandafter
         \expandafter\sa@@getdate\csname sa@pkgdate@\pkg\endcsname//00\relax<\sa@@getdate#3//00\relax
          \global\@namedef{sa@pkgdate@\pkg}{#3}%
        \fi
      }%
    \fi
  }%
  \sa@gobbleeol\sa@removepackages^^J%
}
\def\sa@@getdate#1/#2/#3#4#5\relax{#1#2#3#4}
%    \end{macrocode}
% \end{macro}
%
% \begin{macro}{\sa@sortpgflibs}
% Reads the \cs{usepgflibrary} argument and stores it away.
% Trailing newlines are removed.
%    \begin{macrocode}
\def\sa@collpgflibs{}%
\def\sa@sortpgflibs#1{%
  \@for\lib:=#1\do {%
    \@ifundefined{sa@pgflib@\lib}%
      {%
        \expandafter\g@addto@macro\expandafter\sa@collpgflibs\expandafter{\expandafter,\lib}%
        \global\@namedef{sa@pgflib@\lib}{}%
      }%
      {}%
  }%
  \sa@gobbleeol\sa@removepgflibs^^J%
}
%    \end{macrocode}
% \end{macro}
%
% \begin{macro}{\sa@sorttikzlibs}
% Reads the \cs{usetikzlibrary} argument and stores it away.
% Trailing newlines are removed.
%    \begin{macrocode}
\def\sa@colltikzlibs{}%
\def\sa@sorttikzlibs#1{%
  \@for\lib:=#1\do {%
    \@ifundefined{sa@tikzlib@\lib}%
      {%
        \expandafter\g@addto@macro\expandafter\sa@colltikzlibs\expandafter{\expandafter,\lib}%
        \global\@namedef{sa@tikzlib@\lib}{}%
      }%
      {}%
  }%
  \sa@gobbleeol\sa@removetikzlibs^^J%
}
%    \end{macrocode}
% \end{macro}
%
% \begin{macro}{\sa@sorttikztiminglibs}
% Reads \cs{usetikztiminglibrary} arguments and stores them away.
% Trailing newlines are removed.
%    \begin{macrocode}
\def\sa@colltikztiminglibs{}%
\newcommand\sa@sorttikztiminglibs[2][]{%
  \@ifnextchar[%]
    {\sa@@sorttikztiminglibs{#1}{#2}}%
    {\sa@@sorttikztiminglibs{#1}{#2}[]}%
}
\def\sa@@sorttikztiminglibs#1#2[#3]{%
  \@for\lib:=#2\do {%
    \@ifundefined{sa@tikztimingopts@\lib}%
      {%
        \expandafter\g@addto@macro\expandafter\sa@colltikztiminglibs\expandafter{\expandafter,\lib}%
        \global\@namedef{sa@tikztimingopts@\lib}{#1}%
        \global\@namedef{sa@tikztimingopt@\lib @}{}%
        \ifx\relax#1\relax\else
          \@for\opt:=#1\do{\global\@namedef{sa@tikztimingopt@\lib @\opt}{}}%
        \fi
      }%
      {%
        \ifx\relax#1\relax\else
          \@for\opt:=#1\do{%
            \@ifundefined{sa@tikztimingopt@\lib @\opt}%
              {%
                \expandafter\g@addto@macro\csname sa@tikztimingopts@\lib\expandafter\endcsname\expandafter{\expandafter,\opt}%
                \global\@namedef{sa@tikztimingopt@\lib @\opt}{}%
              }{}%
          }%
        \fi
      }%
    \ifx\relax#3\relax\else
    \@ifundefined{sa@tikztimingdate@\lib}%
      {\global\@namedef{sa@tikztimingdate@\lib}{#3}}%
      {%
        \begingroup
        \edef\@tempa{{\csname sa@tikztimingdate@\lib\endcsname}{#3}}%
        \expandafter\sa@getlargerdate\@tempa
        \expandafter\xdef\csname sa@tikztimingdate@\lib\endcsname{\sa@thedate}%
        \endgroup
      }%
    \fi
  }%
  \sa@gobbleeol\sa@removetikztiminglibs^^J%
}
%    \end{macrocode}
% \end{macro}
%
%
% \begin{macro}{\sa@gobbleopt}
% Gobbles an optional argument and a potential line endings and then executes the command given by |#1|.
%    \begin{macrocode}
\def\sa@gobbleopt#1[#2]{%
  \@ifnextchar^^J%
    {\sa@gobbleeol{#1}}{#1}%
}
%    \end{macrocode}
% \end{macro}
%
%    \begin{macrocode}
\else
%    \end{macrocode}
%
% \begin{macro}{\sa@scanpackages}
% Scans for |\usepackage|.
%    \begin{macrocode}
\def\sa@scanpackages#1\usepackage{%
  \@ifnextchar\sa@endmarker
    {\@gobble}%
    {\sa@collectpackage}
}
%    \end{macrocode}
% \end{macro}
%
% \begin{macro}{\sa@collectpackage}
% Reads \cs{usepackage} arguments (ignores optional date) and stores it away.
% The options are later passed to the package to avoid option clashes.
%    \begin{macrocode}
\newcommand\sa@collectpackage[2][]{%
  \ifx\relax#1\relax\else
    \g@addto@macro\sa@collopts{\PassOptionsToPackage{#1}{#2}}%
  \fi
  \sa@scanpackages
}
\fi
%    \end{macrocode}
% \end{macro}
%
% \begin{macro}{\sa@collopts}
% Accumulator for collected options. Is executed and cleared at the end of this package.
%    \begin{macrocode}
\def\sa@collopts{}
\AtEndOfPackage{\sa@collopts\let\sa@collopts\relax}
%    \end{macrocode}
% \end{macro}
%
% End of |\ifsa@sortsubpreambles|.
%    \begin{macrocode}
\fi
%    \end{macrocode}
%
% \begin{environment}{standalonepreambles}
% This environment simply adds a group and sets the endline character to a printed newline and the argument character
% |#| as a normal character. The first suppresses |\par|'s in the stored sub-preambles while preserving newlines. The latter
% is required to permit macro arguments in the preambles. Otherwise a |#| is doubled to |##| causing compile errors when the
% sub-preambles are used.
% The |.sta| file is closed after this environment.
%    \begin{macrocode}
\def\standalonepreambles{%
  \begingroup
  \endlinechar=\m@ne
  \@makeother\#%
}
\def\endstandalonepreambles{%
  \endgroup
  \endinput
}
%    \end{macrocode}
% \end{environment}
%
% \begin{environment}{subpreambles}
% This environment rereads the sub-preambles from the |.sta| files and stores it globally under the name
% ``\cs{prevsubpreamble@}\meta{file name}''. If sorting is enabled the sub-preambles are also scanned for
% loaded packages.
%    \begin{macrocode}
\long\gdef\subpreamble#1#2\endsubpreamble{%
  \expandafter\gdef\csname prevsubpreamble@#1\endcsname{#2}%
  \ifsa@sortsubpreambles
    \sa@scanpackages#2\usepackage\sa@endmarker
  \fi
}
\def\endsubpreamble{}%
%    \end{macrocode}
% \end{environment}
%
% If in |print| mode open the |.stp| file.
%    \begin{macrocode}
\ifsa@printsubpreambles
  \immediate\openout\sa@out=\jobname.stp\relax
\else
%    \end{macrocode}
% otherwise:
%
% Process |.sta| file from last run. All changes must be made by own macros which define the value globally.
% Therefore the input is wrapped in a group. Some spaces or special line endings could process typeset content,
% which causes errors inside the preamble. To be on the save side the input `content' is stored in a temp box.
%    \begin{macrocode}
\begingroup
  \setbox\@tempboxa\hbox{%
  \InputIfFileExists{\jobname.sta}{}{\PackageInfo{standalone}{STA file not found!}{}{}}%
  }%
\endgroup
%    \end{macrocode}

% \begin{macro}{\AtBeginDocument}
% At begin of the document the |.sta| file is read again. This time the sub-preamble macros are executed as normal.
% The |standalone| macros are defined to be without effect. If `sorting' is enabled \cs{usepackage} is temporarily
% redefined to ignore any given options, which where already passed (\cs{PassOptionsToPackage}) beforehand.
%    \begin{macrocode}
\AtBeginDocument{%
  \let\subpreamble\@gobble
  \let\endsubpreamble\relax
  \let\standalonepreambles\relax
  \let\endstandalonepreambles\relax
  \ifsa@sortsubpreambles
    \let\sa@orig@usepackage\usepackage
    \let\usepackage\sa@usepackagewithoutoptions
  \fi
  \InputIfFileExists{\jobname.sta}{}{}%
  \ifsa@sortsubpreambles
    \let\usepackage\sa@orig@usepackage
  \fi
  \immediate\openout\sa@out=\jobname.sta\relax
  \immediate\write\sa@out{\string\standalonepreambles}%
}
%    \end{macrocode}
% \end{macro}
%
% \begin{macro}{\AtEndDocument}
% At end of the document write end macro to and close |.sta| file.
%    \begin{macrocode}
\AtEndDocument{%
  \sa@write{\string\endstandalonepreambles}%
  \immediate\closeout\sa@out
}
%    \end{macrocode}
% \end{macro}
%
% End of |\ifsa@printsubpreambles|.
%    \begin{macrocode}
\fi
%    \end{macrocode}
%
% End of |\ifsa@subpreambles|.
%    \begin{macrocode}
\fi
%    \end{macrocode}
%
%
% \subsubsection{Skipping of Sub-Preambles in Main Mode}
%
% This macros make the main document skip all preambles in sub-files.
%
% \begin{macro}{\sa@gobbleeol}
% Gobbles all following line endings (i.e.\ empty lines) and then executes the command given by |#1|.
% Because |\@ifnextchar| ignores spaces this also removes lines with only spaces.
%    \begin{macrocode}
\def\sa@gobbleeol#1^^J{%
  \@ifnextchar^^J%
    {\sa@gobbleeol{#1}}{#1}%
}
%    \end{macrocode}
% \end{macro}
%
% \begin{macro}{\sa@gobbleline}
% Gobbles the rest of the input line. This is required when skipping the body of a file to also skip
% everything on the same line after |\begin{document}|.
%    \begin{macrocode}
\def\sa@gobbleline#1^^J{}%
%    \end{macrocode}
% \end{macro}
%
% \begin{macro}{\standaloneignore}
% \changes{v0.3}{2010/03/26}{New macro}
% This macro must only be used in a sub-file before a |\documentclass|. It gobbles everything up to this macro
% and then executes the |standalone| definition of it shown further below.
% It should be written as |\csname standaloneignore\endcsname| to ignore errors in standalone mode.
% The second definition allows the user to also write |\csname standaloneignore \endcsname| (note the extra space)
% without errors.
%    \begin{macrocode}
\long\def\standaloneignore#1\documentclass{%
  \sa@documentclass
}
\@namedef{standaloneignore\space}{\standaloneignore}
%    \end{macrocode}
% \end{macro}
%
% \begin{macro}{\sa@documentclass}
% The |standalone| definition of \cs{documentclass}. If the sub-preambles are to be processed then the
% starting content is written into the output file etc., but only for the first time this sub-file is included.
% Some input related settings are set-up (line endings, macro argument and comments).
% Finally \cs{sa@gobble} is called to process the preamble.
%    \begin{macrocode}
\newcommand{\sa@documentclass}[2][]{%
  \let\document\sa@document
  \begingroup
  \ifsa@subpreambles
    \@ifundefined{sa@written@\currfilepath}%
      {%
        \ifsa@printsubpreambles
          \ifsa@sortsubpreambles
            \begingroup
              \edef\@tempa{^^J\@percentchar\space Preamble from file '\currfilepath'^^J}%
              \expandafter\g@addto@macro\expandafter\sa@preamble\expandafter{\@tempa}%
            \endgroup
          \else
            \sa@write{^^J\@percentchar\space Preamble from file '\currfilepath'}%
          \fi
        \else
          \sa@write{\string\subpreamble{\currfilepath}}%
        \fi
      }{}%
    \global\@namedef{subpreamble@\currfilepath}{}%
    \ifsa@printsubpreambles
      \endlinechar=`\^^J%
    \else
      \endlinechar=\m@ne
    \fi
    \@makeother\#%
    \@nameuse{sa@percent}%
  \fi
  \def\sa@gobbleto{document}%
  \sa@gobbleeol\sa@gobble^^J%
}
%    \end{macrocode}
% \end{macro}
%
% \begin{macro}{\sa@gobble}
% Gobbles everything to the next |\begin|, then checks if it was a |\begin{document}|.
% If sub-preamble extraction is activated it accumulates the skipped content in macros
% named ``\cs{subpreamble@}\meta{file name}''. Every sub-file is remembered and its preamble is only saved once.
% In |print| mode the file body is ignored and a appropriate warning is printed,
% otherwise the current and previous sub-preamble of the current processed file are compared. If different the file body is also
% ignored to avoid errors due to possible newly required but not loaded packages. The user is warned again about this
% and is asked to rerun \LaTeX{}.
%    \begin{macrocode}
\def\sa@preamble{}%
\long\def\sa@gobble#1\begin#2{%
  \def\@tempa{#2}%
  \ifx\@tempa\sa@gobbleto
    \ifsa@subpreambles
      \expandafter\g@addto@macro\csname subpreamble@\currfilepath\endcsname{#1}%
      \@ifundefined{sa@written@\currfilepath}%
        {%
          \ifsa@printsubpreambles
            \ifsa@sortsubpreambles
              \sa@removepackages#1\usepackage\sa@endmarker
            \else
              \begingroup
              \let\sa@preamble\empty
              \sa@removeonlyifstandalone#1\onlyifstandalone\sa@endmarker
              \expandafter\sa@write\expandafter{\expandafter\unexpanded\expandafter{\sa@preamble}}%
              \endgroup
            \fi
          \else
            \sa@write{\unexpanded{#1}}%
            \sa@write{\string\endsubpreamble}%
          \fi
        }{}%
      \global\@namedef{sa@written@\currfilepath}{}%
      \ifsa@printsubpreambles
        \def\next{%
          \endgroup
          \PackageWarning{standalone}{Running 'standalone' package in sub-preamble print mode. All body content of file `\currfilepath' is ignored!}{}{}%
          \hbox to 1pt{\vbox to 1pt{}}%
          \endinput
          %\sa@gobbleline
        }%
      \else
      \expandafter
      \ifx
      \csname prevsubpreamble@\currfilepath\expandafter\endcsname
      \csname     subpreamble@\currfilepath\endcsname
        \def\next{\expandafter\endgroup\expandafter\begin\expandafter{\sa@gobbleto}}%
      \else
        %\expandafter\show\csname prevsubpreamble@\currfilepath\endcsname
        %\expandafter\show\csname     subpreamble@\currfilepath\endcsname
        \def\next{%
          \endgroup
          \PackageWarning{standalone}{Sub-preamble of file '\currfilepath' has changed. Content will be ignored. Please rerun LaTeX!}{}{}%
          \immediate\write\@mainaux{%
            \@percentchar\space standalone package info: Rerun LaTeX!
          }%
          \hbox to 1pt{\vbox to 1pt{}}%
          \endinput
          %\sa@gobbleline
        }%
      \fi
      \fi
    \else
      \def\next{\expandafter\endgroup\expandafter\begin\expandafter{\sa@gobbleto}}%
    \fi
  \else
    \ifsa@subpreambles
      \expandafter\g@addto@macro\csname subpreamble@\currfilepath\endcsname{#1\begin{#2}}%
      \@ifundefined{sa@written@\currfilepath}%
        {\sa@write{\unexpanded{#1\begin{#2}}}}{}%
    \fi
    \def\next{\sa@gobble}%
  \fi
  \next
}
%    \end{macrocode}
% \end{macro}
%
% \begin{environment}{standalone}
% Provide an empty definition of the |standalone| environment. The class is defining it with the code required in |standalone| mode.
%    \begin{macrocode}
\@ifundefined{standalone}
  {\newenvironment{standalone}[1][]{}{}}
  {}
%    \end{macrocode}
% \end{environment}
%
% \begin{environment}{standalone}
% Provide an `empty' definition of the |standaloneframe| environment.
% It only gobbles all arguments: |<...>[<...>][...]{...}{...}|. Please note that the last two |{ }| arguments are also optional.
% The class is defining it with the code required in |standalone| mode.
%    \begin{macrocode}
\@ifundefined{standaloneframe}
  {\@ifundefined{beamer@newenv}
    {\newenvironment{standaloneframe}[1][]{%
      \@ifnextchar[%]
        {\sa@framegobbleopt}{\sa@framegobbleargs}}{}%
    }
    {\newenvironment<>{standaloneframe}[1][]{%
      \@ifnextchar[%]
        {\sa@framegobbleopt}{\sa@framegobbleargs}}{}%
    }
   \def\sa@framegobbleopt[#1]{\sa@framegobbleargs}
   \def\sa@framegobbleargs{%
     \@ifnextchar\bgroup
       {\sa@framegobbleargs@}%
       {}%
   }
   \def\sa@framegobbleargs@#1{%
    \@ifnextchar\bgroup
      {\@gobble}%
      {}%
   }
  }
  {}
%    \end{macrocode}
% \end{environment}
%
% \begin{macro}{\sa@orig@document}
% \begin{macro}{\sa@orig@enddocument}
% Store original |document| environment.
%    \begin{macrocode}
\let\sa@orig@document\document
\let\sa@orig@enddocument\enddocument
%    \end{macrocode}
% \end{macro}
% \end{macro}
%
% \begin{macro}{\document}
% Redefine the |\begin{document}| of the main file to redefine \cs{documentclass}.
% This can not be done using \cs{AtBeginDocument} because the original redefines
% \cs{documentclass} itself after executing the hook.
%    \begin{macrocode}
\def\document{%
  \sa@orig@document
  \let\documentclass\sa@documentclass
  \ignorespaces
}
%    \end{macrocode}
% \end{macro}
%
% \begin{macro}{\sa@document}
% This is the |\begin{document}| of the sub files. It does nothing except of
% redefining |\end{document}| and calling our own |atbegindocument| hook.
%    \begin{macrocode}
\def\sa@document{%
  \let\enddocument\sa@enddocument
  \sa@atbegindocument
}
%    \end{macrocode}
% \end{macro}
%
% \begin{macro}{\sa@enddocument}
% This is the |\end{document}| of the sub files. It does nothing except of
% calling our own |atenddocument| hook and then the `after end document' handler.
%    \begin{macrocode}
\def\sa@enddocument{%
  \sa@atenddocument
  \aftergroup\sa@@enddocument
}
%    \end{macrocode}
% \end{macro}
%
% \begin{macro}{\sa@@enddocument}
% This is a `after end document' handler for the sub-files. It restores macros and
% ends the input of the file.
%    \begin{macrocode}
\def\sa@@enddocument{%
  %\let\document\sa@orig@document
  \let\enddocument\sa@orig@enddocument
  \endinput
}
%    \end{macrocode}
% \end{macro}
%
% \begin{macro}{\sa@atbegindocument}
% This hook simply ignores all spaces after |\begin{document}| in the sub files.
%    \begin{macrocode}
\def\sa@atbegindocument{%
  \ignorespaces
}%
%    \end{macrocode}
% \end{macro}
%
% \begin{macro}{\sa@atenddocument}
% This hook simply ignores the last skip (normally the spaces) before |\end{document}| in the sub files.
%    \begin{macrocode}
\def\sa@atenddocument{%
  \unskip
}%
%    \end{macrocode}
% \end{macro}
%
%
% \subsubsection{Include Standalone}
%
%    \begin{macrocode}
\RequirePackage{gincltex}
%    \end{macrocode}
%
%
% \begin{macro}{\includestandalone}
%    \begin{macrocode}
\newcommand*\includestandalone[2][]{%
    \begingroup
    \edef\@tempa{{#2\sa@graphicext}}%
    \expandafter\includestandalone@\@tempa{#2}{#1}%
    \endgroup
}
%    \end{macrocode}
% \end{macro}
%
%
% \begin{macro}{\includestandalone@}
%% 0 = PDF if exists, TEX otherwise
%% 1 = force TEX
%% 2 = force PDF
%% 3 = build PDF if not exists
%% 4 = build PDF if older than TEX
%    \begin{macrocode}
\def\includestandalone@#1#2#3{%
    \ifcase\sa@mode
    \relax% 0
        \IfFileExists{#1}%
            {\includegraphics[#3]{#1}}%
            {\includegraphics[#3]{#2.tex}}%
    \or% 1
        \includegraphics[#3]{#2.tex}%
    \or% 2
        \includegraphics[#3]{#1}%
    \or% 3
        \IfFileExists{#1}%
            {\includegraphics[#3]{#1}}%
            {\sa@buildgraphic{#2}%
             \IfFileExists{#1}%
                {\includegraphics[#3]{#1}}%
                {\PackageWarning{standalone}%
                 {Graphic '#1' could not be build.^^J%
                  Shell escape activated?}%
                 \includegraphics[#3]{#2.tex}%
                }%
            }%
    \else% 4
        \IfFileExists{#1}%
            {\sa@comparegranpic{#2}}%
            {\sa@buildgraphic{#2}%
             \IfFileExists{#1}%
                {\includegraphics[#3]{#1}}%
                {\PackageWarning{standalone}%
                    {Graphic '#1' could not be build.^^J%
                    Shell escape activated?}%
                    \includegraphics[#3]{#2.tex}%
                }%
            }%
    \fi
}
%    \end{macrocode}
% \end{macro}
%
%
% \begin{macro}{\sa@comparegranpic}[1]{file base name}
% Compares if the |.tex| file newer than the corresponding graphic file and
% rebuilds the graphic if so.
%    \begin{macrocode}
\def\sa@comparegranpic#1{%
    \begingroup
    \let\next\empty
    \expandafter\sa@comparegranpic@\pdffilemoddate{#1\sa@graphicext}\relax
    \let\sa@picdate\sa@date
    \let\sa@pictime\sa@time
    \expandafter\sa@comparegranpic@\pdffilemoddate{#1.tex}\relax
    \ifnum\sa@date>\sa@picdate\relax
        \def\next{\sa@buildgraphic{#1}}%
    \else
        \ifnum\sa@date=\sa@picdate\relax
            \ifnum\sa@time>\sa@pictime\relax
                \def\next{\sa@buildgraphic{#1}}%
            \fi
        \fi
    \fi
    \expandafter
    \endgroup
    \next
}
%    \end{macrocode}
% \end{macro}
%
%
% \begin{macro}{\sa@comparegranpic@}
% Used to parse the file modification date. 1st stage.
%    \begin{macrocode}
\begingroup
\@makeother\D
\@makeother:
\global\@namedef{sa@comparegranpic@}D:#1#2#3#4#5#6#7#8#9\relax{%
    \def\sa@date{#1#2#3#4#5#6#7#8}%
    \sa@comparegranpic@@#9\relax
}
\endgroup
%    \end{macrocode}
% \end{macro}
%
% \begin{macro}{\sa@comparegranpic@@}
% Used to parse the file modification date. 2nd stage.
%    \begin{macrocode}
\def\sa@comparegranpic@@#1#2#3#4#5#6#7\relax{%
    \def\sa@time{#1#2#3#4#5#6}%
    \def\sa@tz{#7}%
}
%    \end{macrocode}
% \end{macro}
%
%
%
% \begin{macro}{\sa@buildgraphic}
% Compiles the given external file.
% The state of the shell escape is checked.
%    \begin{macrocode}
\def\sa@buildgraphic#1{%
    \ifeof18
        \PackageError{standalone}{Shell escape needed to create graphic! Use the '-shell-escape' option.}{}%
    \else
        \begingroup
        \edef\image{#1}%
        \immediate\write18{\sa@compilecmd}%
        \endgroup
    \fi
}
%    \end{macrocode}
% \end{macro}
%
% \iffalse
%</sty>
% \fi
%
